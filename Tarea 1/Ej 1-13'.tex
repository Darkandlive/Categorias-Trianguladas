\documentclass{article}
\usepackage[utf8]{inputenc}
\usepackage{mathrsfs}
\usepackage[spanish,es-lcroman]{babel}
\usepackage{amsthm}
\usepackage{amssymb}
\usepackage{enumitem}
\usepackage{graphicx}
\usepackage{caption}
\usepackage{float}
\usepackage{amsmath,stackengine,scalerel,mathtools}
\usepackage{xparse, tikz-cd, pgfplots}
\usetikzlibrary{babel}
\usepackage{xstring}
\usepackage{mathrsfs}
\usepackage{comment}
\usepackage[all]{xy}
\usepackage{faktor}


\def\subnormeq{\mathrel{\scalerel*{\trianglelefteq}{A}}}
\newcommand{\properideal}{%
	\mathrel{\ooalign{$\lneq$\cr\raise.22ex\hbox{$\lhd$}\cr}}}
\newcommand{\radofset}[1]{rad\lrprth{#1}}
\newcommand{\spectrumring}[1]{Spec\lrprth{#1}}
\newcommand{\Z}{\mathbb{Z}}
\newcommand{\La}{\mathscr{L}}
\newcommand{\crdnlty}[1]{
	\left|#1\right|
}
\newcommand{\lrprth}[1]{
	\left(#1\right)
}
\newcommand{\lrbrack}[1]{
	\left\{#1\right\}
}
\newcommand{\lrsqp}[1]{
	\left[#1\right]
}
\newcommand{\descset}[3]{
	\left\{#1\in#2\ \vline\ #3\right\}
}
\newcommand{\descapp}[6]{
	#1: #2 &\rightarrow #3\\
	#4 &\mapsto #5#6 
}
\newcommand{\arbtfam}[3]{
	{\left\{{#1}_{#2}\right\}}_{#2\in #3}
}
\newcommand{\arbtfmnsub}[3]{
	{\left\{{#1}\right\}}_{#2\in #3}
}
\newcommand{\fntfmnsub}[3]{
	{\left\{{#1}\right\}}_{#2=1}^{#3}
}
\newcommand{\fntfam}[3]{
	{\left\{{#1}_{#2}\right\}}_{#2=1}^{#3}
}
\newcommand{\fntfamsup}[4]{
	\lrbrack{{#1}^{#2}}_{#3=1}^{#4}
}
\newcommand{\arbtuple}[3]{
	{\left({#1}_{#2}\right)}_{#2\in #3}
}
\newcommand{\fntuple}[3]{
	{\left({#1}_{#2}\right)}_{#2=1}^{#3}
}
\newcommand{\gengroup}[1]{
	\left< #1\right>
}
\newcommand{\stblzer}[2]{
	St_{#1}\lrprth{#2}
}
\newcommand{\cmmttr}[1]{
	\left[#1,#1\right]
}
\newcommand{\grpindx}[2]{
	\left[#1:#2\right]
}
\newcommand{\syl}[2]{
	Syl_{#1}\lrprth{#2}
}
\newcommand{\grtcd}[2]{
	mcd\lrprth{#1,#2}
}
\newcommand{\lsttcm}[2]{
	mcm\lrprth{#1,#2}
}
\newcommand{\amntpSyl}[2]{
	\mu_{#1}\lrprth{#2}
}
\newcommand{\gen}[1]{
	gen\lrprth{#1}
}
\newcommand{\ringcenter}[1]{
	C\lrprth{#1}
}
\newcommand{\zend}[2]{
	End_{\mathbb{Z}}^{#2}\lrprth{#1}
}
\newcommand{\cend}[2]{
	End_{#1}\lrprth{#2}
}
\newcommand{\catmatrix}[4]{Mat_{#1\times #2}\lrprth{#3,#4}}
\newcommand{\genmod}[2]{
	\left< #1\right>_{#2}
}
\newcommand{\genlin}[1]{
	\mathscr{L}\lrprth{#1}
}
\newcommand{\opst}[1]{
	{#1}^{op}
}
\newcommand{\ringmod}[3]{
	\if#3l
	{}_{#1}#2
	\else
	\if#3r
	#2_{#1}
	\fi
	\fi
}
\newcommand{\ringbimod}[4]{
	\if#4l
	{}_{#1-#2}#3
	\else
	\if#4r
	#3_{#1-#2}
	\else 
	\ifstrequal{#4}{lr}{
		{}_{#1}#3_{#2}
	}
	\fi
	\fi
}
\newcommand{\ringmodhom}[3]{
	Hom_{#1}\lrprth{#2,#3}
}
\newcommand{\catarrow}[4]{
	\if #4e
			#1:#2\twoheadrightarrow #3
	\else \if #4m
			#1:#2\hookrightarrow #3
	\else \if #4i
		    #1:#2\tilde{\to} #3
			\else #1:#2\to #3
	\fi
	\fi
	\fi
}
\newcommand{\nattrans}[4]{
	Nat_{\lrsqp{#1,#2}}\lrprth{#3,#4}
}
\ExplSyntaxOn

\NewDocumentCommand{\functor}{O{}m}
{
	\group_begin:
	\keys_set:nn {nicolas/functor}{#2}
	\nicolas_functor:n {#1}
	\group_end:
}

\keys_define:nn {nicolas/functor}
{
	name     .tl_set:N = \l_nicolas_functor_name_tl,
	dom   .tl_set:N = \l_nicolas_functor_dom_tl,
	codom .tl_set:N = \l_nicolas_functor_codom_tl,
	arrow      .tl_set:N = \l_nicolas_functor_arrow_tl,
	source   .tl_set:N = \l_nicolas_functor_source_tl,
	target   .tl_set:N = \l_nicolas_functor_target_tl,
	Farrow      .tl_set:N = \l_nicolas_functor_Farrow_tl,
	Fsource   .tl_set:N = \l_nicolas_functor_Fsource_tl,
	Ftarget   .tl_set:N = \l_nicolas_functor_Ftarget_tl,	
	delimiter .tl_set:N= \_nicolas_functor_delimiter_tl,	
}

\dim_new:N \g_nicolas_functor_space_dim

\cs_new:Nn \nicolas_functor:n
{
	\begin{tikzcd}[ampersand~replacement=\&,#1]
		\dim_gset:Nn \g_nicolas_functor_space_dim {\pgfmatrixrowsep}		
		\l_nicolas_functor_dom_tl
		\arrow[r,"\l_nicolas_functor_name_tl"] \&
		\l_nicolas_functor_codom_tl
		\\[\dim_eval:n {1ex-\g_nicolas_functor_space_dim}]
		\l_nicolas_functor_source_tl
		\xrightarrow{\l_nicolas_functor_arrow_tl}
		\l_nicolas_functor_target_tl
		\arrow[r,mapsto] \&
		\l_nicolas_functor_Fsource_tl
		\xrightarrow{\l_nicolas_functor_Farrow_tl}
		\l_nicolas_functor_Ftarget_tl
		\_nicolas_functor_delimiter_tl
	\end{tikzcd}
}

\ExplSyntaxOff


\newcommand{\limseq}[3]{
	\if#3u
	\lim\limits_{#2\to\infty}#1	
		\else
			\if#3s
			#1\to
			\else
				\if#3w
				#1\rightharpoonup
				\else
					\if#3e
					#1\overset{*}{\to}
					\fi
				\fi
			\fi
	\fi
}

\newcommand{\norm}[1]{
	\crdnlty{\crdnlty{#1}}
}

\newcommand{\inter}[1]{
	int\lrprth{#1}
}
\newcommand{\cerrad}[1]{
	cl\lrprth{#1}
}

\newcommand{\restrict}[2]{
	\left.#1\right|_{#2}
}

\newcommand{\realprj}[1]{
	\mathbb{R}P^{#1}
}

\newcommand{\fungroup}[1]{
	\pi_{1}\lrprth{#1}	
}

\ExplSyntaxOn

\NewDocumentCommand{\shortseq}{O{}m}
{
	\group_begin:
	\keys_set:nn {nicolas/shortseq}{#2}
	\nicolas_shortseq:n {#1}
	\group_end:
}

\keys_define:nn {nicolas/shortseq}
{
	A     .tl_set:N = \l_nicolas_shortseq_A_tl,
	B   .tl_set:N = \l_nicolas_shortseq_B_tl,
	C .tl_set:N = \l_nicolas_shortseq_C_tl,
	f      .tl_set:N = \l_nicolas_shortseq_f_tl,
	g   .tl_set:N = \l_nicolas_shortseq_g_tl,	
	lcr   .tl_set:N = \l_nicolas_shortseq_lcr_tl,
	T .tl_set:N = \l_nicolas_shortseq_T_tl,	
	h .tl_set:N = \l_nicolas_shortseq_h_tl,	
	
	A		.initial:n =X,
	B		.initial:n =Y,
	C		.initial:n =Z,
	f    .initial:n =,
	g   	.initial:n=,
	lcr   	.initial:n=lr,
	T   	.initial:n=T,
	h   	.initial:n=,	
}

\cs_new:Nn \nicolas_shortseq:n
{
	\begin{tikzcd}[ampersand~replacement=\&,#1, row~sep=large, column~sep=large]
		\IfSubStr{\l_nicolas_shortseq_lcr_tl}{l}{0 \arrow{r} \&}{
		\IfSubStr{\l_nicolas_shortseq_lcr_tl}{n}{\dots \arrow{r} \&}{}
		}
		\l_nicolas_shortseq_A_tl
		\arrow{r}{\l_nicolas_shortseq_f_tl} \&
		\l_nicolas_shortseq_B_tl
		\arrow[r, "\l_nicolas_shortseq_g_tl"] \&
		\l_nicolas_shortseq_C_tl
		\IfSubStr{\l_nicolas_shortseq_lcr_tl}{t}{\arrow{r}{\l_nicolas_shortseq_h_tl}\&\l_nicolas_shortseq_T_tl\IfSubStr{\l_nicolas_shortseq_lcr_tl}{n}{}{\l_nicolas_shortseq_A_tl}}{\IfSubStr{\l_nicolas_shortseq_lcr_tl}{m}{\arrow{r}{\l_nicolas_shortseq_h_tl}\&\l_nicolas_shortseq_T_tl}{}}
		\IfSubStr{\l_nicolas_shortseq_lcr_tl}{r}{ \arrow{r} \& 0}{
		\IfSubStr{\l_nicolas_shortseq_lcr_tl}{n}{ \arrow{r} \& \dots}{
	}}
	\end{tikzcd}
}

\ExplSyntaxOff

\ExplSyntaxOn

\NewDocumentCommand{\tmorph}{O{}m}
{
	\group_begin:
	\keys_set:nn {nicolas/tmorph}{#2}
	\nicolas_tmorph:n {#1}
	\group_end:
}

\keys_define:nn {nicolas/tmorph}
{
	A     .tl_set:N = \l_nicolas_tmorph_A_tl,
	B   .tl_set:N = \l_nicolas_tmorph_B_tl,
	C .tl_set:N = \l_nicolas_tmorph_C_tl,
	T .tl_set:N = \l_nicolas_tmorph_T_tl,
	f      .tl_set:N = \l_nicolas_tmorph_f_tl,
	g   .tl_set:N = \l_nicolas_tmorph_g_tl,		
	h .tl_set:N = \l_nicolas_tmorph_h_tl,	
	p .tl_set:N = \l_nicolas_tmorph_p_tl,	
	q .tl_set:N = \l_nicolas_tmorph_q_tl,	
	r .tl_set:N = \l_nicolas_tmorph_r_tl,	
	Ap     .tl_set:N = \l_nicolas_tmorph_Ap_tl,
	Bp   .tl_set:N = \l_nicolas_tmorph_Bp_tl,
	Cp .tl_set:N = \l_nicolas_tmorph_Cp_tl,
	fp      .tl_set:N = \l_nicolas_tmorph_fp_tl,
	gp   .tl_set:N = \l_nicolas_tmorph_gp_tl,		
	hp .tl_set:N = \l_nicolas_tmorph_hp_tl,
	I      .tl_set:N = \l_nicolas_tmorph_I_tl,
	II   .tl_set:N = \l_nicolas_tmorph_II_tl,		
	III .tl_set:N = \l_nicolas_tmorph_III_tl,
	
	A		.initial:n =X,
	B		.initial:n =Y,
	C		.initial:n =Z,
	T   	.initial:n=T,
	p	.initial:n=f,
	q	.initial:n=g,
	r	.initial:n=h,
	f    .initial:n =,
	g   	.initial:n=,
	h   	.initial:n=,	
	Ap		.initial:n =X',
	Bp		.initial:n =Y',
	Cp		.initial:n =Z',
	fp    .initial:n =,
	gp   	.initial:n=,
	hp   	.initial:n=,
	I   	.initial:n=,
	II   	.initial:n=,
	III   	.initial:n=,
}

\cs_new:Nn \nicolas_tmorph:n
{
	\begin{tikzcd}[ampersand~replacement=\&,#1]
		\l_nicolas_tmorph_A_tl
		\arrow{rr}{\l_nicolas_tmorph_f_tl}\arrow{dd}[swap]{\l_nicolas_tmorph_p_tl} \&\&
		\l_nicolas_tmorph_B_tl
		\arrow[rr, "\l_nicolas_tmorph_g_tl"]\arrow{dd}{\l_nicolas_tmorph_q_tl} \&\&
		\l_nicolas_tmorph_C_tl
		\arrow{rr}{\l_nicolas_tmorph_h_tl}\arrow{dd}{\l_nicolas_tmorph_r_tl}\&\& \l_nicolas_tmorph_T_tl\l_nicolas_tmorph_A_tl\arrow{dd}{\l_nicolas_tmorph_T_tl\l_nicolas_tmorph_p_tl}\\
		\& \l_nicolas_tmorph_I_tl\&\& \l_nicolas_tmorph_II_tl\&\& \l_nicolas_tmorph_III_tl\&\\
		\l_nicolas_tmorph_Ap_tl
		\arrow{rr}{\l_nicolas_tmorph_fp_tl} \&\&
		\l_nicolas_tmorph_Bp_tl
		\arrow[rr, "\l_nicolas_tmorph_gp_tl"] \&\&
		\l_nicolas_tmorph_Cp_tl
		\arrow{rr}{\l_nicolas_tmorph_hp_tl}\&\& \l_nicolas_tmorph_T_tl\l_nicolas_tmorph_Ap_tl
	\end{tikzcd}
}

\ExplSyntaxOff

\newcommand{\cc}{\mathscr{C}}
\newcommand{\spmat}[1]{%
	\left(\begin{smallmatrix}#1\end{smallmatrix}\right)%
}
\newcommand{\pushpull}{texto}
\newcommand{\testfull}{texto}
\newcommand{\testdiag}{texto}
\newcommand{\testcodiag}{texto}
\ExplSyntaxOn

\NewDocumentCommand{\commutativesquare}{O{}m}
{
	\group_begin:
	\keys_set:nn {nicolas/commutativesquare}{#2}
	\nicolas_commutativesquare:n {#1}
	\group_end:
}

\keys_define:nn {nicolas/commutativesquare}
{	
	A     .tl_set:N = \l_nicolas_commutativesquare_A_tl,
	B   .tl_set:N = \l_nicolas_commutativesquare_B_tl,
	C .tl_set:N = \l_nicolas_commutativesquare_C_tl,
	D .tl_set:N = \l_nicolas_commutativesquare_D_tl,
	P .tl_set:N = \l_nicolas_commutativesquare_P_tl,
	f      .tl_set:N = \l_nicolas_commutativesquare_f_tl,
	g   .tl_set:N = \l_nicolas_commutativesquare_g_tl,
	h   .tl_set:N = \l_nicolas_commutativesquare_h_tl,
	k .tl_set:N = \l_nicolas_commutativesquare_k_tl,
	l .tl_set:N = \l_nicolas_commutativesquare_l_tl,
	m .tl_set:N = \l_nicolas_commutativesquare_m_tl,
	n .tl_set:N = \l_nicolas_commutativesquare_n_tl,
	pp .tl_set:N = \l_nicolas_commutativesquare_pp_tl,
	up .tl_set:N = \l_nicolas_commutativesquare_up_tl,
	diag .tl_set:N = \l_nicolas_commutativesquare_diag_tl,	
	codiag .tl_set:N = \l_nicolas_commutativesquare_codiag_tl,		
	diaga .tl_set:N = \l_nicolas_commutativesquare_diaga_tl,	
	codiaga .tl_set:N = \l_nicolas_commutativesquare_codiaga_tl,		
	
	A		.initial:n =A,
	B		.initial:n =B,
	C		.initial:n =C,
	D		.initial:n =D,
	P		.initial:n =P,
	f    .initial:n =,
	g    .initial:n =,
	h    .initial:n =,
	k    .initial:n =,
	l    .initial:n =,
	m    .initial:n =,
	n	 .initial:n =,
	pp .initial:n=h,	
	up .initial:n=f,
	diag .initial:n=f,
	codiag .initial:n=f,
	diaga .initial:n=,
	codiaga .initial:n=,
}

\cs_new:Nn \nicolas_commutativesquare:n
{
	\renewcommand{\pushpull}{\l_nicolas_commutativesquare_pp_tl}
	\renewcommand{\testfull}{\l_nicolas_commutativesquare_up_tl}
	\renewcommand{\testdiag}{\l_nicolas_commutativesquare_diag_tl}
	\renewcommand{\testcodiag}{\l_nicolas_commutativesquare_codiag_tl}
	\begin{tikzcd}[ampersand~replacement=\&,#1]
		\if \pushpull h
			\if \testfull t
				\l_nicolas_commutativesquare_P_tl
				\arrow[bend~left]{drr}{\l_nicolas_commutativesquare_l_tl
				}
				\arrow[bend~right,swap]{ddr}{\l_nicolas_commutativesquare_m_tl}\arrow{dr}{\l_nicolas_commutativesquare_n_tl}\& \& \\
			\fi
			\if \testfull t
			\&
			\fi\l_nicolas_commutativesquare_A_tl
			\if \testdiag t
				\arrow{dr}[near~end]{\l_nicolas_commutativesquare_diaga_tl}
			\fi
			\arrow{r}{\l_nicolas_commutativesquare_f_tl} \arrow{d}[swap]{\l_nicolas_commutativesquare_g_tl} \&
			\l_nicolas_commutativesquare_B_tl
			 \arrow{d}{\l_nicolas_commutativesquare_h_tl}
			 \if \testcodiag t
			 \arrow{dl}[near~end,swap]{\l_nicolas_commutativesquare_codiaga_tl}
			 \fi
			 \\
			\if \testfull t
			\&
			\fi\l_nicolas_commutativesquare_C_tl
			\arrow{r}[swap]{\l_nicolas_commutativesquare_k_tl} \& \l_nicolas_commutativesquare_D_tl
		\else \if \pushpull l
			\if \testfull t
			\l_nicolas_commutativesquare_P_tl
			\& \& \\
			\& \arrow{ul}[swap]{\l_nicolas_commutativesquare_n_tl}
			\fi
			\l_nicolas_commutativesquare_A_tl
			  \&
			\l_nicolas_commutativesquare_B_tl
			\if \testfull t
			\arrow[bend~right,swap]{ull}{\l_nicolas_commutativesquare_l_tl
			}
			\fi
			 \arrow{l}[swap]{\l_nicolas_commutativesquare_f_tl}
			 \\
			\if \testfull t
			\& \arrow[bend~left]{uul}{\l_nicolas_commutativesquare_m_tl}
			\fi
			\l_nicolas_commutativesquare_C_tl\arrow{u}{\l_nicolas_commutativesquare_g_tl}
			\if \testcodiag t
			\arrow{ur}[near~start]{\l_nicolas_commutativesquare_codiaga_tl}
			\fi
			  \& \l_nicolas_commutativesquare_D_tl\arrow{l}{\l_nicolas_commutativesquare_k_tl}
			  \if \testdiag t
			  \arrow{ul}[near~start,swap,crossing~over]{\l_nicolas_commutativesquare_diaga_tl}
			  \fi
			\arrow{u}[swap]{\l_nicolas_commutativesquare_h_tl}
			\fi
		\fi
		
	\end{tikzcd}
}

\ExplSyntaxOff

\ExplSyntaxOn

\NewDocumentCommand{\commutativehouse}{O{}m}
{
	\group_begin:
	\keys_set:nn {nicolas/commutativehouse}{#2}
	\nicolas_commutativehouse:n {#1}
	\group_end:
}

\keys_define:nn {nicolas/commutativehouse}
{	
	A     .tl_set:N = \l_nicolas_commutativehouse_A_tl,
	B   .tl_set:N = \l_nicolas_commutativehouse_B_tl,
	C .tl_set:N = \l_nicolas_commutativehouse_C_tl,
	D .tl_set:N = \l_nicolas_commutativehouse_D_tl,
	E .tl_set:N = \l_nicolas_commutativehouse_E_tl,
	f      .tl_set:N = \l_nicolas_commutativehouse_f_tl,
	g   .tl_set:N = \l_nicolas_commutativehouse_g_tl,
	h   .tl_set:N = \l_nicolas_commutativehouse_h_tl,
	k .tl_set:N = \l_nicolas_commutativehouse_k_tl,
	l .tl_set:N = \l_nicolas_commutativehouse_l_tl,
	m .tl_set:N = \l_nicolas_commutativehouse_m_tl,
	diag .tl_set:N = \l_nicolas_commutativehouse_diag_tl,	
	codiag .tl_set:N = \l_nicolas_commutativehouse_codiag_tl,		
	diaga .tl_set:N = \l_nicolas_commutativehouse_diaga_tl,	
	codiaga .tl_set:N = \l_nicolas_commutativehouse_codiaga_tl,		
	
	A		.initial:n =A,
	B		.initial:n =B,
	C		.initial:n =C,
	D		.initial:n =D,
	E		.initial:n =E,
	f    .initial:n =f,
	g    .initial:n =g,
	h    .initial:n =h,
	k    .initial:n =k,
	l    .initial:n =l,
	m    .initial:n =m,	
	diag .initial:n=f,
	codiag .initial:n=f,
	diaga .initial:n=,
	codiaga .initial:n=,
}

\cs_new:Nn \nicolas_commutativehouse:n
{
	\begin{tikzcd}[ampersand~replacement=\&,#1]
		\& \l_nicolas_commutativehouse_A_tl  \arrow{dr}{\l_nicolas_commutativehouse_g_tl} \& \\
		\l_nicolas_commutativehouse_B_tl \arrow{ur}{\l_nicolas_commutativehouse_f_tl}\arrow{rr}{\l_nicolas_commutativehouse_h_tl}\arrow{d}[swap]{\l_nicolas_commutativehouse_k_tl} \& \&  \l_nicolas_commutativehouse_C_tl\arrow{d}{\l_nicolas_commutativehouse_l_tl} \\
		\l_nicolas_commutativehouse_D_tl\arrow[swap]{rr}{\l_nicolas_commutativehouse_m_tl}\& \& \l_nicolas_commutativehouse_E_tl
	\end{tikzcd}
}

\ExplSyntaxOff

\ExplSyntaxOn

\NewDocumentCommand{\snakelem}{O{}m}
{
	\group_begin:
	\keys_set:nn {nicolas/snakelem}{#2}
	\nicolas_snakelem:n {#1}
	\group_end:
}

\keys_define:nn {nicolas/snakelem}
{	
	A     .tl_set:N = \l_nicolas_snakelem_A_tl,
	B   .tl_set:N = \l_nicolas_snakelem_B_tl,
	C .tl_set:N = \l_nicolas_snakelem_C_tl,
	Ap .tl_set:N = \l_nicolas_snakelem_Ap_tl,
	Bp .tl_set:N = \l_nicolas_snakelem_Bp_tl,
	Cp      .tl_set:N = \l_nicolas_snakelem_Cp_tl,
	AtoB	.tl_set:N = \l_nicolas_snakelem_AtoB_tl,
	BtoC   .tl_set:N = \l_nicolas_snakelem_BtoC_tl,
	AptoBp   .tl_set:N = \l_nicolas_snakelem_AptoBp_tl,
	BptoCp .tl_set:N = \l_nicolas_snakelem_BptoCp_tl,
	AtoAp .tl_set:N = \l_nicolas_snakelem_AtoAp_tl,
	BtoBp .tl_set:N = \l_nicolas_snakelem_BtoBp_tl,
	CtoCp .tl_set:N = \l_nicolas_snakelem_CtoCp_tl,	
	full .tl_set:N = \l_nicolas_snakelem_full_tl,	
	
	A		.initial:n =A,
	B		.initial:n =B,
	C		.initial:n =C,
	Ap			.initial:n =A',
	Bp		.initial:n =B',
	Cp    .initial:n =C',
	AtoB    .initial:n =,
	BtoC    .initial:n =,
	AptoBp    .initial:n =,
	BptoCp    .initial:n =,
	AtoAp    .initial:n =,	
	BtoBp .initial:n=,
	CtoCp .initial:n=,
	full  .initial:n=f,
}

\cs_new:Nn \nicolas_snakelem:n
{
	\begin{tikzcd}[ampersand~replacement=\&,#1]
		 \if \l_nicolas_snakelem_full_tl t
		 	0\arrow{r}
		 \fi \&\l_nicolas_snakelem_A_tl\arrow{r}{\l_nicolas_snakelem_AtoB_tl}\arrow{d}[swap]{\l_nicolas_snakelem_AtoAp_tl}\& \l_nicolas_snakelem_B_tl\arrow{r}{\l_nicolas_snakelem_BtoC_tl}\arrow{d}{\l_nicolas_snakelem_BtoBp_tl}\& \l_nicolas_snakelem_C_tl\arrow{r}\arrow{d}{\l_nicolas_snakelem_CtoCp_tl} \& 0\\
		 0\arrow{r}\& \l_nicolas_snakelem_Ap_tl\arrow{r}[swap]{\l_nicolas_snakelem_AptoBp_tl}\&\l_nicolas_snakelem_Bp_tl\arrow{r}[swap]{\l_nicolas_snakelem_BptoCp_tl}\& \l_nicolas_snakelem_Cp_tl\if \l_nicolas_snakelem_full_tl t
		 \arrow{r}\& 0
		 \fi
	\end{tikzcd}
}

\ExplSyntaxOff

\newcommand{\redhomlgy}[2]{
	\tilde{H}_{#1}\lrprth{#2}
}
\newcommand{\nilrad}[1]{nil\lrprth{#1}}
\newcommand{\ringofpoly}[2]{#1\lrsqp{#2}}
\newcommand{\ringunits}[1]{U\lrprth{#1}}
\newcommand{\affspace}[2]{\mathds{A}^{#1}\lrprth{#2}}
\newcommand{\copyandpaste}{t}
\newcommand{\moncategory}[2]{Mon_{#1}\lrprth{-,#2}}
\newcommand{\epicategory}[2]{Epi_{#1}\lrprth{#2,-}}
\theoremstyle{definition}
\newtheorem{define}{Definición}
\newtheorem*{definesn}{Definición}
\newtheorem{lem}{Lema}
\newtheorem*{lemsn}{Lema}
\newtheorem{teor}{Teorema}
\newtheorem*{teosn}{Teorema}
\newtheorem{prop}{Proposición}
\newtheorem*{propsn}{Proposición}
\newtheorem{coro}{Corolario}
\newtheorem*{obs}{Observación}

\title{Categorías trianguladas\\ \large Ejercicios 1-13'}
\author{Luis Gerardo Arruti Sebastian\\ Sergio Rosado Zúñiga\\Eduardo León Rodríguez}
\date{}
\begin{document}
	\maketitle
	\begin{enumerate}[label=\textbf{Ej \arabic*.}]
		%Ej 1
		\item Sean $\mathscr{C}$ una categoría aditiva y $\catarrow{T}{\mathscr{C}}{\mathscr{C}}{}$ un funtor de traslación. Se tiene que:
		\begin{enumerate}[label=(\textit{\alph*})]
			\item $\mathscr{T}\lrprth{\mathscr{C},T}$ es una categoría.
			\item $\catarrow{\varphi=(f,g,h)}{\eta}{\mu}{}$ es un isomorfismo en $\mathscr{T}\lrprth{\mathscr{C},T}$ si y sólo si $f,g$ y $h$ son isomorfismos en $\mathscr{C}$.
		\end{enumerate} 
		\begin{proof}
			\boxed{(a)} \underline{C1}. Por definición se tiene que \begin{equation*}
						Hom\lrprth{\mathscr{T}\lrprth{\mathscr{C},T}}=\bigcup_{\lrprth{\eta,\mu}\in {Obj\lrprth{\mathscr{T}\lrprth{\mathscr{C},T}}}^2}\ringmodhom{\mathscr{T}\lrprth{\mathscr{C},T}}{\eta}{\mu}.
			\end{equation*}
		Sea $\lrprth{\eta,\mu}\in {Obj\lrprth{\mathscr{T}\lrprth{\mathscr{C},T}}}^2$, con $\eta=\lrprth{X,Y,Z,u,v,w}$ y $\mu=\lrprth{X',Y',Z',u',v',w'}$. Se tiene que 
		\begin{align*}
			\ringmodhom{\mathscr{T}\lrprth{\mathscr{C},T}}{\eta}{\mu}\subseteq\ringmodhom{\mathscr{C}}{X}{X'}\times\ringmodhom{\mathscr{C}}{Y}{Y'}\times\ringmodhom{\mathscr{C}}{Z}{Z'},
		\end{align*}
		donde $\ringmodhom{\mathscr{C}}{X}{X'}, \ringmodhom{\mathscr{C}}{Y}{Y'}, \ringmodhom{\mathscr{C}}{Z}{Z'}$ son conjuntos, por lo cual $\ringmodhom{\mathscr{T}\lrprth{\mathscr{C},T}}{\eta}{\mu}$ también lo es.\\
		
		\underline{C2}. Se tiene por la definición de los morfismos de triangulos.\\
		
		\underline{C3} Sean
		\begin{align*}
			\eta&=\shortseq{lcr=t,f=u,g=v,h=w,},\\ \mu&=\shortseq{lcr=t,A=A,B=B,C=C,f=r,g=s,h=t,},\\
			\nu&=\shortseq{lcr=t,A=L,B=M,C=N,f=o,g=p,h=q,}
			\intertext{y}
			\varphi&=\lrprth{f,g,h}\in\ringmodhom{\mathscr{T}\lrprth{\mathscr{C},T}}{\eta}{\mu},\\
			\psi&=\lrprth{i,j,k}\in\ringmodhom{\mathscr{T}\lrprth{\mathscr{C},T}}{\mu}{\nu}.
		\end{align*}
		 \begin{enumerate}[label=(\roman*)]
			\item Verificaremos primeramente que la composición de morfismos de triangulos está bien definida. 
			Se tiene que 
			\begin{align*}
				\tmorph{f=u,g=v,h=w,Ap=A,Bp=B,Cp=C,T=T,fp=r,gp=s,hp=t,p=f,q=g,r=h,I=\text{(I)},II=\text{(II)},III=\text{(III)},}
				\intertext{y}
				\tmorph{A=A,B=B,C=C,T=T,f=r,g=s,h=t,Ap=L,Bp=M,Cp=N,fp=o,gp=p,hp=q,p=i,q=j,r=k,I=\text{(IV)},II=\text{(V)},III=\text{(VI)},}
			\end{align*}
			son diagramas conmutativos en $\mathscr{C}$. Así
			\begin{align*}
				\lrprth{jg}u&=j\lrprth{gu}=j\lrprth{rf},&&\text{por (I)}\\
				&=\lrprth{jr}f=\lrprth{oi}f,&&\text{por (IV)}\\
				&=o\lrprth{if}.
			\end{align*}
			En forma análoga, empleando (II) y (IV), y (III) y (VI) respectivamente, se verifica que 
			\begin{align*}
				\lrprth{kh}v&=p\lrprth{jg}\\
				T\lrprth{if}w&=\lrprth{Ti Tf}w=q\lrprth{kh}.
			\end{align*}
			Con lo cual $\psi\circ\varphi= \lrprth{if,jg,kh}\in\ringmodhom{\mathscr{T}\lrprth{\mathscr{C},T}}{\eta}{\nu}$ y así la correspondencia $\circ$ está bien definida; más aún es asociativa, puesto que la composición en $\mathscr{C}$ lo es.
			\item Dado que $T1_A=1_{TA}$, se tiene que
			\begin{equation*}
				\tmorph{A=A,B=B,C=C,f=u,g=v,h=w,Ap=A,Bp=B,Cp=C,fp=u,gp=v,hp=w,p=1_X,q=1_Y,r=1_Z,}
			\end{equation*}
			es un diagrama conmutativo en $\mathscr{C}$ y así \begin{equation*}
				\chi:=\lrprth{1_A,1_B,1_C}\in\ringmodhom{\mathscr{T}\lrprth{\mathscr{C},T}}{\mu}{\mu};
			\end{equation*} más aún, dado que $1_A, 1_B, 1_C$ son las respectivas identidades en $\mathscr{C}$ para los objetos $A, B$ y $C$, se tiene que $\chi\varphi=\varphi$ y $\psi\xi=\psi$.
		\end{enumerate}
			\boxed{(b)} Continuaremos usando las descripciones dadas para los triangulos $\eta$ y $\mu$ dadas al comienzo de ($a$).\\
			
			Se tiene que $\exists\ \xi\in\ringmodhom{\mathscr{T}\lrprth{\mathscr{C},T}}{\mu}{\eta}$, con $\xi=\lrprth{f',g',h'}$, tal que 
			\begin{align*}
				\lrprth{f'f,g'g,h'h}&=\xi\varphi=1_{\eta}=\lrprth{1_X,1_Y,1_Z}
				\intertext{y} 
				\lrprth{ff',gg',hh'}&=\varphi\xi=1_{\mu}=\lrprth{1_A,1_B,1_C},
			\end{align*}
			de lo cual se sigue que $f, g$ y $h$ son isomorfismos en $\mathscr{C}$ con inversa, respectivamente, $f', g'$ y $h'$.\\
			
			$\underline{\impliedby}$ Notemos que bajo esta hipótesis se tiene que
			\begin{align*}
				\lrprth{f^{-1}f,g^{-1}g,h^{-1}h}=\lrprth{1_X,1_Y,1_Z},
				\lrprth{ff^{-1},gg^{-1},hh^{-1}}=\lrprth{1_A,1_B,1_C},
			\end{align*}
			por lo cual, por la definición de la ocmposición de morfismos de triángulos, basta con verificar que $\lrprth{f^{-1},g^{-1},h^{-1}}\in\ringmodhom{\mathscr{T}\lrprth{\mathscr{C},T}}{\mu}{\eta}$.\\
			Como $\varphi\in\ringmodhom{\mathscr{T}\lrprth{\mathscr{C},T}}{\mu}{\eta}$, entonces
			\begin{equation*}
				\tmorph{Ap=A,Bp=B,Cp=C,f=u,g=v,h=w,fp=r,gp=s,hp=t,p=f,q=g,r=h,}
			\end{equation*}
			es un diagrama conmutativmo en $\mathscr{C}$, con lo cual
			\begin{align*}
				gu&=rf,\\
				hv&=sg,\\
				Tfw&=th.
				\intertext{Por lo anterior y dado que, por ser $T$ un funtor, $T$ manda isomorfismos en isomorfismos y más aún $\lrprth{Tf}^{-1}=T\lrprth{f^{-1}}$, se tiene que}
				uf^{-1}&=g^{-1}r,\\
				vg^{-1}&=h^{-1}s,\\
				wh^{-1}&=T\lrprth{f^{-1}}t.
			\end{align*}
			Por lo tanto
			\begin{equation*}
				\tmorph{A=A,B=B,C=C,f=r,g=s,h=t,fp=u,gp=v,hp=w,p=f^{-1},q=g^{-1},r=h^{-1},Ap=X,Bp=Y,Cp=Z,}
			\end{equation*}
			es un diagrama conmutativo en $\mathscr{C}$ y así se tiene lo deseado.\\
		\end{proof}
		%Ej 2
		\item Sean $(\mathscr{C},T,\triangle)$ una categoría pre-triangulada y \xymatrix{X\ar[r]^u & Y\ar[r]^v & Z\ar[r]^w & TX } \\en $\triangle$.
		Pruebe que $u\in SKer (v),\quad  v\in SCoKer(u)\cap SKer(w)$\,\, y \\ $w\in SCoKer(v)$.
		\begin{proof}
			Primero se probará que $u\in SKer (v)$. \\Por el teorema ( 1.2.a ) se tiene que $vu=0$.\\
			
			Sea $r:M\to Y$ tal que $vr=0$. Como $M\in \mathscr{C}$, entonces por el teorema ( TR1a ) \xymatrix{M\ar[r]^{1_M} & M\ar[r] & 0\ar[r] & TM } está en 
			$\triangle$, y por ( TR2 ) se puede rotar el triangulo de las hipótesis tal se tiene el siguiente diagrama:\\
			
			\centerline{
				\xymatrix{
					M\ar[r]^{1_M}\ar[d]_r &0\ar[d]_0\ar[r]&T(M)\ar[r]^{-T(1_M)}&TM\ar[d]^{T(r)}\\ 
					Y\ar[r]_{v}&Z\ar[r]_w&T(x)\ar[r]_{-T(u)}& T(y)\,.
				}
			}
			
			Así, por ( TR3 ) existe $s':T(M)\to T(X)$ tal que \\ $(T(r))(-T(1_M))=(-T(u))(s')$, y como $T$ es autofuntor, entonces \\
			$-T(r\circ 1_M))=-T(u\circ T^{-1}(s'))$. Si se toma $s= T^{-1}(s'):M\to X$  se tiene que $r=us$, por lo tanto $u\in SKer(v)$.\\
			
			Veamos ahora que $v\in SCoker(u)$.\\
			
			Anteriormente se observó que $vu=0$. Sea $t:Y\to M$ tal que $tu=0$. Como $M\in \mathscr{C}$, entonces 
			por el teorema ( TR1a ) \xymatrix{M\ar[r]^{1_M} & M\ar[r] & 0\ar[r] & TM } \\está en $\triangle$, así por ( TR2 ) 
			\xymatrix{0=T^{-1}(0)\ar[r]_0 &M\ar[r]_{1_M} & M\ar[r] & 0 } está en $\triangle$.\\
			
			Además, como $tu=0$ entonces entonces se tiene el siguiente diagrama:\\
			
			\centerline{
				\xymatrix{
					X\ar[r]^u\ar[d]^0 & Y\ar[r]^v\ar[d]^t & Z\ar[r]^w & TX \ar[d]^{T(0)=0}\\
					0\ar[r]_0 &M\ar[r]_{1_M} & M\ar[r] & 0.
				}
			}
			
			Así, por ( TR3 ) existe $s:Z\to M$ tal que $1_M\circ t=sv$ y por lo tanto $v\in SCoKer(u)$.\\
			
			Por último, rotando el triángulo de las hipótesis por ( TR1c ) se tiene que \xymatrix{Y\ar[r]_{v}&Z\ar[r]_w&T(x)\ar[r]_{-T(u)}& T(y)} está en $\triangle$, 
			así por lo demostrado $v\in Ker(w)$ y $w\in CoKer(v)$.
			
		\end{proof}
		%Ej 3
		\item Sean $(\mathscr{C},T,\triangle)$ una categoría pre-triangulada y $(f,g,h):\eta \to \mu$ en $T(\mathscr{C},T)$, con $\eta ,\mu \in \triangle$. Pruebe que si dos de los tres morfismos $f,g$ y $h$ son isos, entonces el tercero también lo es.
		
		\begin{proof}
			\begin{enumerate}
				\item Caso 1. Si $f$ y $g$ son isos en $\mathscr{C}$, entonces por el teorema 1.2(c) se concluye que $h$ es iso en $\mathscr{C}$.
				
				\item Suponga que $f$ y $h$ son isos en $\mathscr{C}$. Se inicia con la siguiente situaci\'on:
				
				\begin{center}
					\begin{tikzcd}
						X \arrow{r}{u} \arrow{d}{f} & Y \arrow{r}{v} \arrow{d}{g} & Z \arrow{r}{w} \arrow{d}{h} & TX \arrow{d}{Tf} \\
						A \arrow{r}{a} & B \arrow{r}{b} & C \arrow{r}{c} & TA \\
					\end{tikzcd}
				\end{center}
				
				después de aplicar una rotación a la izquierda (por 1.3) a los triángulos $\eta$ y $\mu$, se obtiene la siguiente situaci\'on:
				
				\begin{center}
					\begin{tikzcd}
						T^{-1}Z \arrow{r}{-T^{-1}w} \arrow{d}[swap]{T^{-1}h} & X \arrow{r}{u} \arrow{d}{f} & Y \arrow{r}{v} \arrow{d}{g} & Z \arrow{d}{h} \\
						T^{-1}C \arrow{r}{-T^{-1}c} & A \arrow{r}{a} & B \arrow{r}{b} & C \\
					\end{tikzcd}
				\end{center}
				dado que $-T^{-1}c \circ T^{-1}h=-(T^{-1}c \circ T^{-1}h)=-(f\circ T^{-1}w)=f\circ -T^{-1}w$ y como $T^{-1}h$ es un iso en $\mathscr{C}$ pues $h$ lo es, entonces por el teorema $1.2$ (c) se concluye que $g$ es un iso en $\mathscr{C}$. 
				
				\item Suponga que $g$ y $h$ son isos en $\mathscr{C}$. Una vez mas se tiene la siguiente configuración inicial:
				
				\begin{center}
					\begin{tikzcd}
						X \arrow{r}{u} \arrow{d}{f} & Y \arrow{r}{v} \arrow{d}{g} & Z \arrow{r}{w} \arrow{d}{h} & TX \arrow{d}{Tf} \\
						A \arrow{r}{a} & B \arrow{r}{b} & C \arrow{r}{c} & TA \\
					\end{tikzcd}
				\end{center}
				
				después de aplicar una rotaci\'on a la derecha (por TR2) a los triángulos $\eta$ y $\mu$, se obtiene la siguiente situaci\'on:
				
				\begin{center}
					\begin{tikzcd}
						Y \arrow{r}{v} \arrow{d}{g} & Z \arrow{r}{w} \arrow{d}{h} & TX \arrow{d}{Tf} \arrow{r}{-Tu} & TY \arrow{d}{Tg} \\
						B \arrow{r}{b} & C \arrow{r}{c} & TA \arrow{r}{-Ta} & TB \\
					\end{tikzcd}
				\end{center}
				
				observe que $Tg \circ Tu=Ta\circ Tf$ por lo que $Tg \circ -Tu=-Ta\circ Tf$, dado que $g$ y $h$ son isos, se cuenta con las hipótesis necesarias para concluir que $Tf$ es un iso en $\mathscr{C}$ y como $T^{-1}$ es funtor (es decir preserva isos) se sigue que $f=T^{-1}(Tf)$ es un iso en $\mathscr{C}$. 
			\end{enumerate}
		\end{proof}
		%Ej 4
		\item Sean $\lrprth{\mathscr{C},T,\Delta}$ una categoría pretriangulada y $\eta$ un triangulo distinguido \shortseq{lcr=t,f=u,g=v,h=w,}. Entonces los siguientes triangulos son distinguidos: 
		\begin{enumerate}[label=(\textit{\alph*})]
			\item $\mu=$\shortseq{lcr=t,f=-u,g=-v,h=w,},
			\item $\nu=$\shortseq{lcr=t,f=-u,g=v,h=-w,}, 
			\item $\chi=$\shortseq{lcr=t,f=u,g=-v,h=-w,}.
		\end{enumerate}
		\begin{proof}
		Por ser $T$ en partícular un funtor aditivo se tiene que $T\lrprth{-1_X}=-1_{TX}$, con lo cual
		\begin{align*}
			\tmorph{Ap=X,Bp=Y,Cp=Z,p=-1_X,q=1_Y,r=-1_Z,f=u,g=v,h=w,fp=-u,gp=-v,hp=w,}
		\end{align*}
		es un diagrama conmutativo en $\mathscr{C}$, cuyas columnas son isomorfismos. Por lo tanto por Ej. 1($b$) $\catarrow{\lrprth{-1_X,1_Y,-1_Z}}{\eta}{\mu}{}$ es un isomorfismo en $\mathscr{T}\lrprth{\mathscr{C},T}$ y así, como $\eta\in\Delta$, por TR1($b$)  $\mu\in\Delta$. Análogamente, empleando que 
		\begin{align*}
			\tmorph{Ap=X,Bp=Y,Cp=Z,p=1_X,q=-1_Y,r=-1_Z,f=u,g=v,h=w,fp=-u,gp=v,hp=-w,}
			\intertext{y}
			\tmorph{Ap=X,Bp=Y,Cp=Z,p=-1_X,q=-1_Y,r=1_Z,f=u,g=v,h=w,fp=u,gp=-v,hp=-w,}
		\end{align*}
		son diagramas conmutativos en $\mathscr{C}$, se verifica que $\nu,\chi\in\Delta$.\\
		\end{proof}
		%Ej 5
		\item Sea $(\mathscr{C},T,\triangle)$ una categoría pretriangulada (respectivamente triangulada), pruebe que el triple 
		$(\mathscr{C}^{op},\tilde{T},\tilde{\triangle})$ es una categoría pretriangulada (respectivamente triangulada), donde 
		$\tilde{T}(f^{op})=(T^{-1}(f))^{op}$ y $\tilde{\triangle}$ se define como sigue:\\
		\centerline{
			\xymatrix{
				X\ar[r]^{u^{op}}&Y\ar[r]^{v^{op}}&Z\ar[r]^{w^{op}}&\tilde{T}X&\in \tilde{\triangle}\\
				&&\iff\\
				Z\ar[r]^u&Y\ar[r]^v&X\ar[r]^{Tw}&TZ&\in \triangle\,.
			}
		}\\
		
		En tal caso se define $(\mathscr{C},T,\triangle)^{op}:=(\mathscr{C}^{op},\tilde{T},\tilde{\triangle}).$\\
		Esto es, $(\mathscr{C}^{op},\tilde{T},\tilde{\triangle})$ es una categoría pretriangulada (respectivamente triangulada) opuesta de
		$(\mathscr{C},T,\triangle)^{op}$.
		
		\begin{proof}
			
			Se puede observar que al tomar  $(\mathscr{C}^{op},\tilde{T},\tilde{\triangle})$ como se describe en las hipótesis, al ser $\mathscr{C}$ 
			una categoría abeliana, entonces $\mathscr{C}^{op}$ también es una categoría abeliana, donde la operación está definida por \\
			$f^{op}\tilde{+}g^{op}:=(f+g)^{op}$ para cada $f,g\in Mor(\mathscr{C})$ y\\
			$+: Mor(\mathscr{C})\times Mor(\mathscr{C})\rightarrow Mor(\mathscr{C})$ la operación definida en $ \mathscr{C}$.\\
			
			Por lo anterior se tiene entonces que el funtor opuesto de una categoría aditiva cualquiera $\mathscr{A}$ es aditivo, pues si $f,g\in 
			\operatorname{Hom}_{\mathscr{A}}(A,B)$ entonces 
			\[D_{\mathscr{A}}(f+g)=(f+g)^{op}=f^{op}\tilde{+}g^{op}=D_{\mathscr{A}}(f)\tilde{+}D_{\mathscr{A}}(g).\]
			
			Con esto en mente se demostrará que $\tilde{T}$ es un autofuntor aditivo.\\
			
			Lo primero que se tiene que notar es que $\tilde{T}=D_{\mathscr{C}}T^{-1}D_{\mathscr{C}^{op}}$, por lo que $\tilde{T}$ es un funtor aditivo al ser 
			composición de funtores aditivos. Por otra parte se tiene que $\tilde{G}=D_{\mathscr{C}}T^{}D_{\mathscr{C}^{op}}$ es un funtor aditivo, y es tal 
			que 
			\begin{gather*}
				\tilde{G}\tilde{T}=D_{\mathscr{C}}T^{}D_{\mathscr{C}^{op}}D_{\mathscr{C}}T^{-1}D_{\mathscr{C}^{op}}\\
				=D_{\mathscr{C}}TT^{-1}D_{\mathscr{C}^{op}}\\
				=D_{\mathscr{C}}D_{\mathscr{C}^{op}}\\
				=1_{\mathscr{C}^{op}}.
			\end{gather*}
			
			\begin{gather*}
				\tilde{T}\tilde{G}=D_{\mathscr{C}}T^{-1}D_{\mathscr{C}^{op}}D_{\mathscr{C}}T^{}D_{\mathscr{C}^{op}}\\
				=D_{\mathscr{C}}T^{-1}TD_{\mathscr{C}^{op}}\\
				=D_{\mathscr{C}}D_{\mathscr{C}^{op}}\\
				=1_{\mathscr{C}^{op}}.
			\end{gather*}
			
			Por lo tanto $\tilde{T}$ es un autofuntor aditivo.\\
			
			Vemos ahora que $(\mathscr{C}^{op},\tilde{T},\tilde{\triangle})$ es una categoría pretriangulada.\\
			
			\boxed{TR1 a)} Sea $X\in \mathscr{C}^{op}$ entonces $X\in \mathscr{C}$, como $(\mathscr{C},T,\triangle)$ es pretriangulada, entonces 
			\xymatrix{X\ar[r]^{1_X}&X\ar[r]&0\ar[r]&TX\in \triangle}.\\ Rotando a la izquierda dos veces por ( 1.3 ) sobre $\mathscr{C}$ se tiene que \\
			\xymatrix@+2pc{T^{-1}X\ar[r]^{T^{-1}(0)}&0\ar[r]&X\ar[r]^{1_X}&X\in \tilde{\triangle}}.\\ Así, por definición de $\tilde{\triangle}$ y por el 
			hecho de que $T^{-1}X=\tilde{T}X$ se tiene que\\
			\xymatrix@+2pc{X\ar[r]^{(1_X)^{op}=1_X}&X\ar[r]&0\ar[r]&\tilde{T}X\in \tilde{\triangle}}.\\
			
			\boxed{TR1 b)} Sean $\alpha:\xymatrix{C\ar[r]^{w^{op}}&B\ar[r]^{v^{op}}&A\ar[r]^{u^{op}}&\tilde{T}C}\quad $ y \\
			$\beta:\xymatrix{Z\ar[r]^{t^{op}}&Y\ar[r]^{s^{op}}&X\ar[r]^{r^{op}}&\tilde{T}Z}$
			
			en $\mathscr{C}^{op}$, tal que $\alpha\in \tilde{\triangle}$ y $\alpha\cong \beta$. Entonces se tienen isomorfismos \\
			$\varphi^{op}, \psi^{op}, \theta^{op}\in Mor(\mathscr{C}^{op})$ tales que el siguiente diagrama conmuta en $\mathscr{C}^{op}$:\\
			
			\centerline{
				\xymatrix{
					\alpha:&C\ar[d]^{\varphi^{op}}\ar[r]^{w^{op}}&B\ar[d]^{\psi^{op}}\ar[r]^{v^{op}}&A\ar[d]^{\theta^{op}}\ar[r]^{u^{op}}&
					\tilde{T}C\ar[d]^{\tilde{T}(\varphi^{op})}\\
					\beta:&Z\ar[r]_{t^{op}}&Y\ar[r]_{s^{op}}&X\ar[r]_{r^{op}}&\tilde{T}Z\,.
				}
			}
			Así el siguiente diagrama conmuta:\\
			\centerline{
				\xymatrix{
					\alpha':&T^{-1}C\ar[r]^{-u}&A\ar[r]^{v}&B\ar[r]^{w}&C\\
					\beta':&T^{-1}Z\ar[u]^{T^{-1}(\varphi)}\ar[r]_{-r}&X\ar[u]^{\theta^{}}\ar[r]_{s}&Y\ar[u]^{\psi^{}}\ar[r]_{t}&Z\ar[u]^{\varphi^{}}\,,
				}
			}
			
			pues \begin{itemize} 
				\item[$\bullet$)] $-u\circ T^{-1}(\varphi)=-[( T^{-1}(\varphi))^{op}\circ u^{op}]^{op}=-[\tilde{T}(\varphi^{op})\circ u^{op}]^{op}\\
				=-[r^{op}\theta^{op}]^{op}=-[(\theta r)^{op}]^{op}=-[\theta r]=\theta\circ (-r).$
				\item[$\bullet$)] $v\theta=[\theta^{op}v^{op}]^{op}=[s^{op}\psi^{op}]^{op}=[( \psi s)^{op}]^{op}=\psi s$.
				\item[$\bullet$)] $w\psi=[\psi^{op}w^{op}]^{op}=[t^{op}\varphi^{op}]^{op}=[(\varphi t)^{op}]^{op}=\varphi t$.
			\end{itemize}
			
			Como \xymatrix{A\ar[r]^v&B\ar[r]^w&C\ar[r]^{T(u)\quad}&TA\in \triangle} por estar $\alpha\in \tilde{\triangle}$, entonces \\
			$\alpha'\in \triangle$ por ( TR2 ) sobre $\mathscr{C}$ al ser su rotación a izquierda. 
			Así se tiene que $\alpha'\cong \beta'$ con $\alpha'\in \triangle$ y, por ( TR1 b) ), $\beta'\in \triangle$.
			Eso implica que (al rotar $\beta'$ por ( TR2 ) ) el triangulo \xymatrix{X\ar[r]^s&Y\ar[r]^t&Z\ar[r]^{T(r)}&TX\in \triangle} \\y por definición entonces 
			$\beta\in\tilde{\triangle}$.\\
			
			\boxed{TR1 c)} Sea $f^{op}:B\to A$ en $\mathscr{C}^{op}$ entonces $f:A\to B$ está en $\mathscr{C}$. \\
			
			Por ( TR1 c) ) sobre $\mathscr{C}$, existe \xymatrix{B\ar[r]^{\alpha}&Z\ar[r]^\beta&TX} en $\mathscr{C}$ tal que \\
			\xymatrix{A\ar[r]^{f}&B\ar[r]^{\alpha}&Z\ar[r]^\beta&TX\in \triangle}, así por ( TR2 ) sobre $\mathscr{C}$ se tiene que 
			\xymatrix@+3pc{T^{-1}Z\ar[r]^{-T^{-1}(\beta)}&A\ar[r]^{f}&B\ar[r]^{\alpha=T(T^{-1}(\alpha))\quad,}&Z\in \triangle}.
			
			Por lo tanto \xymatrix@+2pc{B\ar[r]^{f^{op}}&A\ar[r]^{-(T^{-1}(\beta))^{op}}&T^{-1}Z\ar[r]^{(T^{-1}(\alpha))^{op}}&\tilde{T}B\in \tilde{\triangle}}
			y así \xymatrix@+2pc{B\ar[r]^{f^{op}}&A\ar[r]^{-(\tilde{T}(\beta)^{op})}&\tilde{T}Z\ar[r]^{\tilde{T}(\alpha)^{op}}&\tilde{T}B\in \tilde{\triangle}}.\\
			
			\boxed{TR2} Sea \xymatrix{X\ar[r]^{u^{op}}&Y\ar[r]^{v^{op}}&Z\ar[r]^{w^{op}}&TX\in \tilde{\triangle}}, entonces por definición 
			\xymatrix{Z\ar[r]^{v}&Y\ar[r]^{u}&X\ar[r]^{T(w)\quad\,\,}&TZ\in {\triangle}}.\\
			
			Por el ejercicio 4 se tiene que  \xymatrix{Z\ar[r]^{v}&Y\ar[r]^{-u}&X\ar[r]^{-T(w)\quad\quad}&TZ\in {\triangle}}, y por \\( 1.3 ) 
			\xymatrix@+3pc{T^{-1}X\ar[r]^{-(T^{-1}(-T(w)))}&Z\ar[r]^{v}&Y\ar[r]^{-u\quad\quad}&X\in {\triangle}}, \\es decir, 
			\xymatrix{T^{-1}X\ar[r]^{w}&Z\ar[r]^{v}&Y\ar[r]^{-u}&X\in {\triangle}}.\\
			
			Entonces por definición de $\tilde{\triangle}$ se tiene que \\
			\centerline{
				\xymatrix@+2pc{Y\ar[r]^{v^{op}}&Z\ar[r]^{w^{op}}&T^{-1}X\ar[r]^{-\tilde{T}(u^{op})}&\tilde{T}Y\in \tilde{\triangle}}
			}\\
			es decir, 
			\xymatrix@+1pc{Y\ar[r]^{v^{op}}&Z\ar[r]^{w^{op}}&\tilde{T}X\ar[r]^{-\tilde{T}(u^{op})\quad}&\tilde{T}Y\in \tilde{\triangle}}.\\
			
			\boxed{TR3} Sean $\eta^{op}=(X,Y,Z,u^{op},v^{op},w^{op})$ \,,\,
			$\mu^{op}=(X_0,Y_0,Z_0,u_0^{op},v_0^{op},w_0^{op})$ en $\tilde{\triangle}$, y $f^{op}:X\to X_0\,,\,g^{op}:Y\to Y_0$ tales que 
			$g^{op}u^{op}=u^{op}_0f^{op}$. Entonces se tiene el siguiente diagrama conmutativo en $\mathscr{C}^{op}$:\\
			
			\centerline{
				\xymatrix{
					\eta:& X\ar[d]^{f^{op}}\ar[r]^{u^{op}}&Y\ar[d]^{g^{op}}\ar[r]^{v^{op}}&Z\ar[r]^{w^{op}}&\tilde{T}X\ar[d]^{\tilde{T}(f^{op})}\\
					\mu:&X_0\ar[r]_{u_0^{op}}&Y_0\ar[r]_{v_0^{op}}&Z_0\ar[r]_{w_0^{op}}&\tilde{T}X_0\,.
				}
			}
			y en consecuencia se tiene el siguiente diagrama conmutativo en $\mathscr{C}$ \\
			
			\centerline{
				\xymatrix{
					Z_0\ar[r]^{v_0}&Y_0\ar[d]^{g}\ar[r]^{u_0}&X_0\ar[d]^{f}\ar[r]\ar[r]^{T(w_0)}&TZ_0&\in \triangle\\
					Z\ar[r]_{v}&Y\ar[r]_{u}&X\ar[r]_{T(w)}&TZ&\in \triangle\,\ldots (1)
				}
			}
			
			donde, como $g^{op}u^{op}=u_0^{op}f^{op}$ entonces $(ug)^{op}=(fu_0)^{op}$, es decir, \\$ug=fu_0$.\\
			
			Rotando a la izquierda por ( 1.3 ), se tiene el siguiente diagrama:\\
			\centerline{
				\xymatrix{
					Y_0\ar[d]^{g}\ar[r]^{u_0}&X_0\ar[d]^{f}\ar[r]\ar[r]^{T(w_0)}&TZ_0\ar[r]^{-T(v_0)}&TY_0\ar[d]^{T(f)}\\
					Y\ar[r]_{u}&X\ar[r]_{T(w)}&TZ\ar[r]_{-T(v)}&TY.
				}
			}
			Así por ( TR3 ) sobre $\triangle$, se tiene que existe $h_1:TZ_0\to TZ$ tal que hace conmutar el diagrama, es decir, $h_1T(w_0)=t(w)f$ y \\
			$-T(v)h_1=T(f)(-T(v_0))$.\\
			
			Tomando $h=T^{-1}(h_1)$ se tiene que, como $T$ es fiel y pleno,
			\begin{gather*}
				-T(v)h_1=-T(gv_0)\\
				T(vh)=T(gv_0)\\
				vh=gv_0
			\end{gather*}
			y
			\begin{gather*}
				h_1T(w_0)=t(w)f\\
				T(h)T(w_0)=T(w)f\,.
			\end{gather*}
			
			Es decir, (1) con el morfismo $h$ es un diagrama conmutativo, y tomando $h^{op}$ se tiene que $v^{op}g^{op}=(gv)^{op}=(vh)^{op}=h^{op}v^{op}$
			y \\ $w^{op}h^{op}=(hw_0)^{op}=(T-1(T(h)T(w_0)))^{op}=(T-1(T(h)f))^{op}=(wT^{-1}(f))^{op}\\
			=(T^{-1}(f))^{op}w^{op}=\tilde{T}(f^{op})w^{op}$.
			Por lo que es diagrama de las hipótesis para TR3 es conmutativo con $h^{op}$.\\
			
			\boxed{TR4} Por último, en caso de ser $(\mathscr{C},T,\triangle)$ categoría triangulada supongamos que tenemos el siguiente diagrama en 
			$\mathscr{C}^{op}$\\
			
			\centerline{
				\xymatrix@+1pc{
					\alpha^{op}:&X\ar@^{=}[d]\ar[r]^{u^{op}}&Y\ar[d]^{v^{op}}\ar[r]^{i^{op}}&Z'\ar[r]^{i'^{op}}&\tilde{T}X\ar@^{=}[d] \\
					\beta^{op}:&X\ar[d]_{u^{op}}\ar[r]^{v^{op}u^{op}}&Z\ar@^{=}[d]\ar[r]^{k^{op}}&Y'\ar[r]^{k'^{op}}&\tilde{T}X\ar[d]^{\tilde{T}(u^{op})}\\
					\gamma^{op}:&Y\ar[r]^{v^{op}}&Z\ar[r]^{r^{op}}&X'\ar[d]^{h^{op}}\ar[r]^{j^{op}}&\tilde{T}Y\ar[d]^{\tilde{T}(i^{op})}\\
					&&&TZ'\ar@^{=}[r]&TZ'&\ldots (1).
				}
			}
			
			Se afirma que existen $f^{op},g^{op}$ tales que \xymatrix{\theta^{op}: Z'\ar[r]^{f^{op}}&Y'\ar[r]^{g^{op}}&X'\ar[r]^{h^{op}\quad}&\tilde{T}
				Z'\in \tilde{\triangle}}\\ y hacen conmutar el diagrama anterior. \\
			
			Observemos el siguiente diagrama en $\mathscr{C}$
			
			\centerline{
				\xymatrix@+1pc{
					\gamma_0:&T^{-1}Z\ar@^{=}[d]\ar[r]^{-T^{-1}(v)}&T^{-1}Y\ar[d]^{T^{-1}(u)}\ar[r]^{-j}&X'\ar[r]^{r}&Z\ar@^{=}[d]\\
					\beta_0:&T^{-1}Z\ar[d]_{T^{-1}(v)}\ar[r]^{-T^{-1}(uv)}&T^{-1}X\ar@^{=}[d]\ar[r]^{-k'}&Y'\ar[r]^{k}&Z\ar[d]^{v}\\
					\alpha_0:&T^{-1}Y\ar[r]^{-T^{-1}(u)}&T^{-1}X\ar[r]^{-i'}&Z'\ar[d]^{T(h)}\ar[r]^{i}&Y\ar[d]^{T(j)}\\
					&&&TX'\ar@^{=}[r]&TX'&\ldots (2).
				}
			}
			
			Observemos que $h^{op}=\tilde{T}(i^{op})j^{op}=(T^{-1}(i))^{op}j^{op}$ entonces $h=jT^{-1}(i)$ y así $T(h)=T(j)i$.\\
			
			Ahora, consideremos a $\alpha,\beta$ y $\gamma$ como los triangulos distinguidos en $\mathscr{C}$ dados por los triangulos en $\tilde{\triangle}$ 
			dados en el primer diagrama:
			\begin{gather*}
				\xymatrix{\alpha:Z'\ar[r]^{i}&Y\ar[r]^{u}&X\ar[r]^{T(i')}&TZ'}\\
				\xymatrix{\beta:Y'\ar[r]^{k}&Z\ar[r]^{uv}&X\ar[r]^{T(k')}&TY'}\\
				\xymatrix{\gamma:X'\ar[r]^{r}&Z\ar[r]^{v}&Y\ar[r]^{T(j)}&TX'}\,,
			\end{gather*}
			
			y rotando dos veces a la izquierda cada uno por ( TR2 ) sobre $\triangle$ se obtienen $\alpha_0,\beta_0$ y $\gamma_0$ respectivamente, los cuales por 
			definición serán elementos de $\tilde{\triangle}$.\\
			
			Como $(\mathscr{C},T,\triangle)$ es categoría triangulada, por el axioma del octaedro existen $g:X'\to Y'$ y $f:Y'\to Z'$ tales que hacen conmutar
			el diagrama (2) y \xymatrix{\theta:X'\ar[r]^{g}&Y'\ar[r]^{f}&Z'\ar[r]^{T(h)\quad\quad}&TX'\in \triangle}. Así\\
			\xymatrix{\theta^{op}:Z'\ar[r]^{f^{op}}&Y'\ar[r]^{g^{op}}&X'\ar[r]^{h^{op}\quad}&TZ'\in \tilde{\triangle}}, 
			y $f^{op},g^{op}$ hacen conmutar el diagrama (1), pues:
			\begin{itemize}
				\item[$\bullet$] $f^{op}i^{op}=(if)^{op}=(vk)^{op}=k^{op}v^{op}$\,.
				\item[$\bullet$] $-i'=f\circ (-k')$ entonces $(-i')^{op}=(f\circ (-k'))^{op}=-(k')^{op}f^{op}$\\ así $i^{op}=k'^{op}f^{op}$.
				\item[$\bullet$] $g^{op}k^{op}=(kg)^{op}=(r)^{op}$.
				\item[$\bullet$] $j^{op}g^{op}=(gj)^{op}=(kT^{-1}(u))^{op}=(T^{-1}(u))^{op}k^{op}=\tilde{T}(u^{op})k^{op}$.
			\end{itemize}
			
		\end{proof}
		%Ej 6
		\item Sean $(\mathscr{C},T,\triangle)$ una categoría pre-triangulada y \xymatrix{\eta:X\ar[r]^u & Y\ar[r]^v & Z\ar[r]^w & TX } \\en $\triangle$. Pruebe que para cada $i\in \mathbb{Z}$, el siguiente triangulo es distinguido
		
		\begin{center}
			\begin{tikzcd}
				\eta^{i}:T^{i}(X) \arrow{r}{(-1)^{i}T^{i}(u)} & T^{i}(Y) \arrow{r}{(-1)^{i}T^{i}(v)} & T^{i}(Z) \arrow{r}{(-1)^{i}T^{i}(w)} & T^{i+1}(X).
			\end{tikzcd}
		\end{center}
		
		\begin{proof}
			En el caso que $i\in \mathbb{N}$ la demostración se sigue por inducción.
			
			\bigskip
			
			Caso base. $i=0$. En este caso $\eta^{i}=\eta^{0}=\eta \in \triangle$.
			
			Hipótesis de inducción. Sea $i>0$ y suponga que $\eta^{i}\in \triangle$.
			
			Paso inductivo. Después de aplicar 3 rotaciones consecutivas a la derecha (TR2) al tri\'angulo $\eta^{i}$ se obtiene el siguiente triangulo distinguido:
			
			\begin{center}
				\begin{tikzcd}
					T^{i+1}(X) \arrow{rr}{(-1)^{i+1}T^{i+1}u} & & T^{i+1}(Y) \arrow{rr}{(-1)^{i+1}T^{i+1}v} & & T^{i+1}(Z) \arrow{rr}{(-1)^{i+1}T^{i+1}w} & & T^{i+2}(X)
				\end{tikzcd}
			\end{center}
			es decir, $\eta^{i+1}\in \triangle$.
			
			\bigskip
			
			Resta demostrar que se encuentran los triángulos del tipo $\eta^{(-i)}$ para $i\geq 0$. Y para demostrar esto se procede también por inducción sobre $i$. El caso base coincide con el demostrado antes, se sigue pues con:  
			
			\bigskip
			
			hipótesis de inducción. Sea $i>0$ y suponga que $\eta^{(-i)}\in \triangle$. 
			
			\bigskip
			
			Paso inductivo. Después de aplicar 3 rotaciones consecutivas a la izquierda (por 1.3) al triangulo $\eta^{(-i)}$ se obtiene el siguiente triangulo distinguido:
			
			\begin{center}
				\begin{tikzcd}
					T^{-(i+1)}(X) \arrow{rr}{(-1)^{-(i+1)}T^{-(i+1)}u} & & T^{-(i+1)}(Y) \arrow{rr}{(-1)^{-(i+1)}T^{-(i+1)}v} & & T^{-(i+1)}(Z) \arrow{rr}{(-1)^{-(i+1)}T^{-(i+1)}w} & & T^{-i}(X)
				\end{tikzcd}
			\end{center}
			es decir, $\eta^{-(i+1)}\in \triangle$. Se concluye el ejercicio.\\
		\end{proof}
		%Ej 7
		\item Sean $\lrprth{\mathscr{C},T\Delta}$ una categoría pretriangulada, $\mathscr{A}$ una categoría abeliana, $\catarrow{F}{\mathscr{C}}{\mathscr{A}}{}$ un funtor cohomológico y, para cada $i\in\mathbb{Z}$, $F^i:=F\circ T^i$. Entonces $\forall$ \shortseq{lcr=t,f=u,g=v,h=w,}$\in\Delta$, se tienen las siguientes sucesiones exactas largas en $\mathscr{A}$ según corresponda a la varianza de $F$:
		\begin{enumerate}[label=(\textit{\alph*})]
			\item \shortseq{lcr=tn,A=F^iX,B=F^iY,C=F^iZ,T=F^{i+1}X,f=F^iu,g=F^iv,h=F^iw,}, si $F$ es covariante;
			\item \shortseq{lcr=tn,A=F^iZ,B=F^iY,C=F^iX,T=F^{i+1}Z,h=F^iu,g=F^iv,f=F^iw,}, si $F$ es contravariante.
		\end{enumerate}
		\begin{proof}
			Comenzaremos verificando lo siguiente:
			\begin{lemsn}
				Sean $\mathscr{A}$ una categoría abeliana y $\catarrow{\alpha}{A}{B}{}$ en $\mathscr{A}$, entonces \begin{enumerate}[label=(\roman*)]
					\item $Ker\lrprth{\alpha}=Ker\lrprth{-\alpha}$,
					\item $Im\lrprth{\alpha}=Im\lrprth{-\alpha}$.
				\end{enumerate}
			\end{lemsn}
			\begin{proof}
				\underline{(i)} Notemos que la afirmación es equivalente a que $\catarrow{\beta}{C}{A}{m}$ es un kernel para $\alpha$ si y sólo si lo es para $-\alpha$.\\ 
				\boxed{\implies} Se tiene que
				\begin{align*}
					\lrprth{-\alpha}\beta&=-\lrprth{\alpha\beta}=0.
				\end{align*}
			Ahora, si $\catarrow{\gamma}{C'}{A}{}$ es tal que $\lrprth{-\alpha}\gamma=0$, entonces $-\lrprth{\alpha\gamma}=0$ y por tanto $\alpha\gamma=0$. Así, por la propiedad universal del kernel aplicada a $\beta$ y $\alpha$ se obtiene que $\exists !$ $\catarrow{\delta}{C'}{C}{}$ tal que
			 $\gamma=\beta\delta$. Con lo cual se he verificado que $\beta$ es un kernel para $-\alpha$.\\
			 \boxed{\impliedby} Como $\beta$ es un kernel para $\alpha':=-\alpha$ entonces por lo probado en \boxed{\implies} $\beta$ es un kernel para $-\alpha'=-\lrprth{-\alpha}=\alpha$.
			 
			 \underline{ii} Primeramente notemos que lo demostrado en (i) garantiza que $Coker\lrprth{\alpha}=Coker\lrprth{-\alpha}$. En efecto, como $\opst{\mathscr{A}}$ al serlo $\mathscr{A}$, se tiene que
			 \begin{align*}
			 	\beta\in Coker\lrprth{\alpha}\text{ en }\mathscr{A}&\iff 			 	
			 	\opst{\beta}\in Ker\lrprth{\opst{\alpha}}\text{ en }\opst{\mathscr{A}},\\
			 	&\iff \opst{\beta}\in Ker\lrprth{\opst{-}\opst{\alpha}}\text{ en }\opst{\mathscr{A}}, && \text{(i)}\\
			 	&\iff \opst{\beta}\in Ker{\opst{\lrprth{-\alpha}}}\text{ en }\opst{\mathscr{A}},\tag{*}\label{penulteq}\\
			 	&\iff {\beta}\in Coker\lrprth{-\alpha}\text{ en }{\mathscr{A}}.	
			 \end{align*}
			La equivalencia dada en (\ref{penulteq}) se sigue de que la estructura aditiva en $\ringmodhom{\opst{\mathscr{A}}}{B}{A}$ viene dada por \begin{align*}
  		 		\opst{\varphi}\opst{+}\opst{\theta}&:=\opst{\varphi+\theta}.
  		 	\end{align*}
			 Ahora, sea $\nu$ un subobjeto de $B$. Dado que $\mathscr{A}$ es abeliana por por 1.6.4 de las notas de Homología relativa en categorías abelianas se tiene que 
			 \begin{align*}
			 	\nu\in Im\lrprth{-\alpha}&\iff \nu\in Ker\lrprth{Coker\lrprth{-\alpha}},\\
			 	&\iff \nu\in Ker\lrprth{Coker\lrprth{\alpha}},\\
			 	&\iff \nu\in Im\lrprth{\alpha},
			 \end{align*}
		 con lo cual se tiene lo deseado.\\
			\end{proof}
			\boxed{(a)} Sea $i\in \mathbb{Z}$. Por Ej. 6
			\begin{equation*}
				\eta:\shortseq{lcr=m,A=T^iX,B=T^iY,C=T^iZ,T=T^{i+1}X,f=\lrprth{-1}^iT^iu,g=\lrprth{-1}^iT^iv,h=\lrprth{-1}^iT^iw,}\in\Delta.
			\end{equation*}
			Así, si suponemos que $F$ es covariante, se tiene que
			\begin{equation*}
				\shortseq{lcr=c,A=FT^iX,B=FT^iY,C=FT^iZ,f=\lrprth{-1}^iT^iu,g=\lrprth{-1}^iT^iv,}
			\end{equation*}
		es una sucesión exacta en $\mathscr{C}$. Por lo anterior y el Lema se sigue que
		\begin{equation*}
			Ker\lrprth{F^iv}=Ker\lrprth{\lrprth{-1}^iF^iv}=Im\lrprth{\lrprth{-1}^iF^iu}=Im\lrprth{F^iu},
		\end{equation*}
		y así
		\begin{equation*}
			\shortseq{lcr=c,A=F^iX,B=F^iY,C=FT^iZ,f=F^iu,g=F^iv,}
		\end{equation*}
		es exacta en $\mathscr{C}$.\\
		Ahora, por (TR2) aplicado a $\eta$ y por ser $T$ un funtor aditivo se tiene que
		\begin{equation*}
			\shortseq{lcr=m,A=T^iY,B=T^iZ,C=T^{i+1}X,T=T^{i+1}Y,f=\lrprth{-1}^iT^iv,g=\lrprth{-1}^iT^iw,h=\lrprth{-1}^{i+1}T^{i+1}u}\in\Delta
		\end{equation*}
		con lo cual, valiéndose nuevamente del Lema y que $F$ es cohomológico se obtiene la siguiente sucesión exacta en $\mathscr{C}$
		\begin{equation*}
			\shortseq{lcr=c,A=F^iY,B=F^iZ,C=F^{i+1}X,f=F^iv,g=F^{i}w,}.
		\end{equation*}
		En forma análoga, aplicando $F$ al triángulo distinguido, obtenido a partir de aplicar dos veces (TR2) a $\eta$,
		\begin{equation*}
			\shortseq{lcr=m,A=T^iZ,B=T^{i+1}X,C=T^{i+1}Y,T=T^{i+1}Z,f=\lrprth{-1}^iT^iw,g=\lrprth{-1}^{i+1}T^{i+1}u,h=\lrprth{-1}^{i+1}T^{i+1}v,}\in\Delta
		\end{equation*}
		se obtiene que
		\begin{equation*}
			\shortseq{lcr=c,A=F^iZ,B=F^{i+1}X,C=F^{i+1}Y,f=F^{i}w,g=F^{i+1}u,}.
		\end{equation*}
		la siguiente sucesión exacta en $\mathscr{C}$.\\
		La arbitrareidad de $i$ nos da lo deseado.\\
		
		\boxed{(b)} Se demuestra en forma análoga a ($a$).\\
		\end{proof}		
		%Ej 8
		\item Sean $\mathscr{C}$ una categoría y $h:A\to B$ en $\mathscr{C}$. Pruebe que: \\
		\centerline{
			\xymatrix{
				\operatorname{Hom}_{\mathscr{C}}(\bullet,h):\operatorname{Hom}_{\mathscr{C}}(\bullet,A)\ar[r]^{\,\,\quad\quad \sim}&
				\operatorname{Hom}_{\mathscr{C}}(\bullet,B)}
			\xymatrix{
				\iff h:A\ar[r]^{\quad\,\,\, \sim}&B.
			}
		}
		\begin{proof}
			Supongamos $h:A\to B$ es isomorfismo y sea $M\in \mathscr{C}$, entonces 
			\xymatrix{
				\operatorname{Hom}_{\mathscr{C}}(M,h):\operatorname{Hom}_{\mathscr{C}}(M,A)\ar[r]&
				\operatorname{Hom}_{\mathscr{C}}(M,B)}. \\Sean $g: B\to A$ tal que $h\circ g=1_B, g\circ h=1_A$ y $r\in \operatorname{Hom}_{\mathscr{C}}(M,A)$
			entonces \\$\operatorname{Hom}_{\mathscr{C}}(M,h)(r)=h\circ r$, así 
			\xymatrix{
				\operatorname{Hom}_{\mathscr{C}}(\bullet,g):\operatorname{Hom}_{\mathscr{C}}(\bullet,B)\ar[r]&
				\operatorname{Hom}_{\mathscr{C}}(\bullet,A)} \\es tal que 
			\[ \operatorname{Hom}_{\mathscr{C}}(M,g)\circ\operatorname{Hom}_{\mathscr{C}}(M,h)(r)=g(hr)=(gh)r=r
			\]
			así $\operatorname{Hom}_{\mathscr{C}}(M,g)\circ\operatorname{Hom}_{\mathscr{C}}(M,h)=1_A$.\\
			
			Análogamente si $s\in \operatorname{Hom}_{\mathscr{C}}(M,B)$ entonces \\
			$\operatorname{Hom}_{\mathscr{C}}(M,h)\circ\operatorname{Hom}_{\mathscr{C}}(M,g)(s)=h(gs)=(hg)s=s$, es decir\\
			$\operatorname{Hom}_{\mathscr{C}}(M,h)\circ\operatorname{Hom}_{\mathscr{C}}(M,g)=1_B$ y así $\operatorname{Hom}_{\mathscr{C}}(M,h)$ es iso.
			\\
			
			Por otra parte, si $\operatorname{Hom}_{\mathscr{C}}(\bullet,h)$ es isomorfismo, entonces el siguiente diagrama conmuta\\
			\centerline{
				\xymatrix@C+1pc{
					\operatorname{Hom}_{\mathscr{C}}(A,A)\ar[d]_{\operatorname{Hom}_{\mathscr{C}}(1_A,A)}
					\ar[r]_{\sim}^{\operatorname{Hom}_{\mathscr{C}}(A,h)} &
					\operatorname{Hom}_{\mathscr{C}}(A,B)\ar[d]^{\operatorname{Hom}_{\mathscr{C}}(1_A,B)}\\
					\operatorname{Hom}_{\mathscr{C}}(A,A)
					\ar[r]^{\sim}_{\operatorname{Hom}_{\mathscr{C}}(A,h)} & \operatorname{Hom}_{\mathscr{C}}(A,B).
				}
			}
			mas aún, como $\operatorname{Hom}_{\mathscr{C}}(A,h)(1_A)=h\circ 1_A=h\in \operatorname{Hom}_{\mathscr{C}}(A,B)$ entonces
			$\operatorname{Hom}^{-1}_{\mathscr{C}}(A,h) h=1_A$.\\
			
			Ahora, por el lema de Yoneda existe $g:B\to A$ tal que \\$\operatorname{Hom}^{-1}_{\mathscr{C}}(A,h)= 
			\operatorname{Hom}_{\mathscr{C}}(A,g)$ en particular \\
			$h\circ g=\operatorname{Hom}^{-1}_{\mathscr{C}}(A,h)(g)=
			\operatorname{Hom}_{\mathscr{C}}(A,h)\operatorname{Hom}^{-1}_{\mathscr{C}}(A,h)(1_B) =1_B$ y \\
			$g\circ h=\operatorname{Hom}^{-1}_{\mathscr{C}}(A,h)\operatorname{Hom}_{\mathscr{C}}(A,h)(1_A)=1_A$ 
			por lo que $h$ es isomorfismo.
			
		\end{proof}
		%Ej 9
		\item Sean $(\mathscr{C},T,\triangle)$ una categoría pre-triangulada y \xymatrix{\eta:X\ar[r]^u & Y\ar[r]^v & Z\ar[r]^w & TX } \\ \xymatrix{\eta':A\ar[r]^a & B\ar[r]^b & C\ar[r]^c & TA } $\in \triangle$. Pruebe que el diagrama en $\mathscr{C}$,
		
		\begin{center}
			\begin{tikzcd}
				X \arrow{r}{u} & Y \arrow{r}{v} \arrow{d}{g} & Z \arrow{r}{w} & T(X) \\
				A \arrow{r}{a} & B \arrow{r}{b} & C \arrow{r}{c} & T(A)
			\end{tikzcd}
		\end{center}
		las siguientes condiciones son equivalentes.
		
		\begin{enumerate}
			\item $bgu=0$
			\item Existe $f:X\to A$ en $\mathscr{C}$ tal que $gu=af$
			\item Existe $h:Z\to C$ en $\mathscr{C}$ tal que $bg=hv$
			\item Existen $f:X\to A$ y $h:Z\to C$ en $\mathscr{C}$ tales que $(f,g,h):\eta \to \eta^{'}$ es un morfismo de triángulos. 
		\end{enumerate}
		Mas a\'un, si $Hom_{\mathscr{C}}(X,T^{-1}C)=0$ y las condiciones anteriores se satisfacen entonces el morfismo $f$ en b) (resp. $h$ en c)) es \'unico.
		
		\begin{proof}
			$a) \Rightarrow b)$. Suponga que $b(gu)=bgu=0$, lo anterior implica que existe $f:X\to A$ tal que $af=gu$ pues $a\in sker(b)$.
			
			\bigskip
			
			$b) \Rightarrow c)$. Suponga que existe $f:X\to A$ en $\cc$ tal que $gu=af$, por TR3 existe $h:Z\to C$ tal que $(f,g,h):\eta \to \eta^{'}$ es un morfismo de triángulos, en particular $bg=hv$.
			
			\bigskip
			
			$c) \Rightarrow d)$. Suponga que existe $h:Z\to C$ en $\cc$ tal que $bg=hv$. Resta estudiar la existencia de alg\'un $f:X\to A$ tal que $gu=af$ y que $T(f)w=ch$. Se inicia con la siguiente configuración:
			
			\begin{center}
				\begin{tikzcd}
					X \arrow{r}{u} & Y \arrow{r}{v} \arrow{d}{g} & Z \arrow{d}{h} \arrow{r}{w} & T(X) \\
					A \arrow{r}{a} & B \arrow{r}{b} & C \arrow{r}{c} & T(A)
				\end{tikzcd}
			\end{center}
			después de aplicar una vez TR2 a los triángulos $\eta$ y $\eta^{'}$ se obtiene la siguiente situación:
			
			\begin{center}
				\begin{tikzcd}
					Y \arrow{r}{v} \arrow{d}{g} & Z \arrow{d}{h} \arrow{r}{w} & T(X) \arrow{r}{-Tu} & T(Y) \arrow{d}{Tg} \\
					B \arrow{r}{b} & C \arrow{r}{c} & T(A) \arrow{r}{-Ta} & T(B) 
				\end{tikzcd}
			\end{center}
			entonces por TR3 y por que $T$ es un automorfismo, existe $f:X\to A$ tal que $T(f)\circ w=c\circ h$ y $T(g)\circ -Tu=-Ta\circ Tf$ por consiguiente $gu=af$. Se puede concluir que $(f,g,h):\eta \to \eta^{'}$ es un morfismo de triángulos.
			
			\bigskip
			
			$d) \Rightarrow a)$. Por hipótesis se sabe que $hv=bg$, es así que se cumple lo siguiente:
			
			\begin{align*}
				bgu =& (bg)u\\
				=& (hv)u\\
				=& h(vu)\\
				=& h0\\
				=& 0
			\end{align*}
			
			A continuación se estudia la unicidad del morfismo $f:X\to A$. Suponga que existe otro morfismo $f':X\to A$ en $\cc$ tal que $gu=af'$. Es así que se tiene la siguiente situación:
			
			\begin{center}
				\begin{tikzcd}
					X \arrow{r}{u} \arrow[d, shift right=2, dashed, swap]{}{f'} \arrow{d}{f} & Y \arrow{r}{v} \arrow{d}{g} & Z \arrow{r}{w} & T(X) \\
					A \arrow{r}{a} & B \arrow{r}{b} & C \arrow{r}{c} & T(A)
				\end{tikzcd}
			\end{center}
			
			después de rotar ambos triángulos una vez a la izquierda (por 1.3) y aplicar el funtor cohomol\'ogico $Hom_{\cc}(X,\_):\cc \to ab$ se tiene el siguiente par de sucesiones exactas:
			
			\begin{center}
				\begin{tikzcd}
					Hom_{\cc}(X,T^{-1}Z) \arrow{d} \arrow{r} & Hom_{\cc}(X,X) \arrow{rr}{Hom_{\cc}(X,u)} \arrow[d, shift right=2, dashed, swap]{}{Hom_{\cc}(X,f')} \arrow{d}{Hom_{\cc}(X,f)} & & Hom_{\cc}(X,Y) \arrow{d}{Hom_{\cc}(X,g)}  \\
					Hom_{\cc}(X,T^{-1}C)=0 \arrow{r} & Hom_{\cc}(X,A) \arrow{rr}[swap]{Hom_{\cc}(X,a)} & & Hom_{\cc}(X,B)
				\end{tikzcd}
			\end{center}
			se puede deducir que $Hom_{\cc}(X,a)$ es un monomorfismo, así pues se cumple lo siguiente:
			
			\begin{align*}
				(Hom_{\cc}(X,a)\circ Hom_{\cc}(X,f))(1_{X}) =& (Hom_{\cc}(X,a)\circ Hom_{\cc}(X,f'))(1_{X})\\
				Hom_{\cc}(X,f)(1_{X}) =& Hom_{\cc}(X,f')(1_{X})\\
				f =& f'
			\end{align*}
			
			como se buscaba.
			
			\bigskip
			
			Para finalizar se estudia la unicidad del morfismo $h:Z\to C$. Suponga que existe otro morfismo $h':Z\to C$ en $\mathcal{C}$ tal que $bg=h'v$. As\'i pues se tiene la siguiente situaci\'on
			
			\begin{center}
				\begin{tikzcd}
					X \arrow{r}{u} \arrow{d}{f} & Y \arrow{r}{v} \arrow{d}{g} & Z \arrow[d, shift right=2, dashed, swap]{}{h'} \arrow{d}{h} \arrow{r}{w} & T(X) \\
					A \arrow{r}{a} & B \arrow{r}{b} & C \arrow{r}{c} & T(A)
				\end{tikzcd}
			\end{center}
			
			despu\'es de rotar ambos triangulos a las izquierda dos veces (por 1.3) y aplicar el funtor contravariante $Hom_{\mathcal{C}}(\_,T^{-1}C):\mathcal{C} \to Ab$ se tiene el siguiente par de sucesiones exactas:
			
			\begin{center}
				\begin{tikzcd}
					Hom_{\cc}(X,T^{-1}C)=0 \arrow{r} & Hom_{\cc}(T^{-1}Z,T^{-1}C) \arrow{rr}{Hom_{\cc}(-T^{-1}v,T^{-1}C)} & & Hom_{\cc}(T^{-1}Y,T^{-1}C)  \\
					Hom_{\cc}(A,T^{-1}C) \arrow{u} \arrow{r} & Hom_{\cc}(T^{-1}C,T^{-1}C) \arrow[u, shift right=2, dashed, swap]{}{Hom_{\cc}(T^{-1}h',T^{-1}C)} \arrow{u}{Hom_{\cc}(T^{-1}h,T^{-1}C)} \arrow{rr}[swap]{Hom_{\cc}(-T^{-1}b,T^{-1}C)} & & Hom_{\cc}(T^{-1}B,T^{-1}C) \arrow{u}
				\end{tikzcd}
			\end{center}
			se puede deducir que $Hom_{\cc}(-T^{-1}v,T^{-1}C)$ es un monomorfismo, as\'i pues se deduce lo siguiente:
			
			\begin{small}
				\begin{align*}
					([-T^{-1}v,T^{-1}C] \circ Hom_{\cc}(T^{-1}h,T^{-1}C))(1_{T^{-1}C}) =& ([-T^{-1}v,T^{-1}C] \circ Hom_{\cc}(T^{-1}h',T^{-1}C))(1_{T^{-1}C})\\
					Hom_{\cc}(T^{-1}h,T^{-1}C))(1_{T^{-1}C})(1_{T^{-1}C}) =& Hom_{\cc}(T^{-1}h',T^{-1}C))(1_{T^{-1}C})(1_{T^{-1}C}) \\
					T^{-1}h =& T^{-1}h'\\
					h =& h'
				\end{align*}
			\end{small}
			donde $[-T^{-1}v,T^{-1}C]=Hom_{\cc}(-T^{-1}v,T^{-1}C)$. 
			
		\end{proof}
		%Ej 10
		\item Sean $\lrprth{\mathscr{C},T,\Delta}$ una categoría pre-triangulada, $\eta:=\shortseq{lcr=t,f=u,g=v,h=w,}$ en $\Delta$ tal que $\ringmodhom{\mathscr{C}}{X}{T^{-1}Z}=0$. Las siguientes condiciones se satisfacen:
		\begin{enumerate}[label=(\textit{\alph*})]
			\item si \begin{equation*}
				\eta'=\shortseq{lcr=t,f=u,g=v',h=w',C=Z'}\in\Delta,
			\end{equation*}
		 entonces $\exists !\ \catarrow{g}{Z}{Z'}{}$ en $\mathscr{C}$ tal que $\catarrow{\lrprth{1_X,1_Y,g}}{\eta}{\eta'}{}$ es un isomorfismo en $\mathscr{T}\lrprth{\mathscr{C},\Delta}$;
			\item si $w'\in\ringmodhom{\mathscr{C}}{Z}{TX}$ es tal que \shortseq{lcr=t,f=u,g=v,h=w',}$\in\Delta$, entonces $w=w'$.
		\end{enumerate}
		\begin{proof}
			\boxed{(a)} Notemos que 
			\begin{center}
				\begin{tikzcd}
					X\arrow{rr}{u}\arrow{dd}[swap]{1_X} & & Y
					\arrow{rr}{v'}\arrow{dd}{1_Y} & &	Z'\arrow{rr}{w'}\ar[dd,dashed, "\exists !\ h"]& & TX\\
					& \text{(I)}& &  & & &\\
					X\arrow{rr}{u} & &Y\arrow{rr}{v} & & Z\arrow{rr}{w}& & TX
				\end{tikzcd}
			\end{center}
		es un diagrama en el cual (I) es un cuadro conmutativo y $\ringmodhom{\mathscr{C}}{X}{T^{-1}Z}=0$, por lo cual aplicando Ej. 9 se sigue que $\exists\ !\ h\in\ringmodhom{\mathscr{C}}{Z'}{Z}$ tal que $\lrprth{f,g,h}$ es un morfismo de triangulos; más aún, por Ej. 3 se tiene que $h$ es un isomorfismo puesto que $f$ y $g$ lo son. De modo que, por Ej. 1($b$), $\catarrow{\lrprth{1_X,1_Y,h}}{\eta}{\eta'}{i}$ en $\mathscr{T}\lrprth{\mathscr{C}, T}$,  con
		\begin{equation*}
			\lrprth{1_X,1_Y,h}^{-1}=\lrprth{\lrprth{1_X}^{-1},\lrprth{1_Y}^{-1},{h}^{-1}}=\lrprth{1_x,1_Y,h^{-1}}.
		\end{equation*}
		Así, tomando $g:=h^{-1}$, se ha verificado la existencia.\\
		
		Sea $g'\in\ringmodhom{\mathscr{C}}{Z}{Z'}$ tal que $\lrprth{1_X,1_Y,g'}$ es un isomorfismo en $\mathscr{T}\lrprth{\mathscr{C}, T}$. Entonces $\catarrow{\lrprth{1_X,1_Y,g^{-1}}\lrprth{1_X,1_Y,g'}^{-1}}{\eta'}{\eta}{i}$ en $\mathscr{T}\lrprth{\mathscr{C}, T}$, luego por la unicidad de $h$ se sigue que $\lrprth{g'}^{-1}=h$ y por lo tanto $g'=h^{-1}$.\\
		
		\boxed{(b)} Sea $\eta'=\lrprth{X,Y,Z,u,v,W'}$. Como $\eta'\in\Delta$ por $(a)$ se tiene que $\exists !\ \catarrow{g}{Z}{Z}{}$ tal que $\catarrow{\lrprth{1_x,1_Y,g}}{\eta}{\eta'}{i}$ en $\mathscr{T}\lrprth{\mathscr{C},T}$, así que
		\begin{equation*}
			\tmorph{Ap=X,Bp=Y,Cp=Z,fp=u,gp=v,hp=w',II=\text{(A)},III=\text{(B)},}
		\end{equation*}
		es un diagrama conmutativo en $\mathscr{C}$. En partícular, por (A) $g$ es un morfismo tal que $gv=v$. Como $1_Zv=v$ y $\ringmodhom{\mathscr{C}}{X}{T^{-1}Z}=0$, por Ej. 9($b$), entonces $g=1_Z$ y así por (B) se tiene que 
		\begin{align*}
			w&=1_{TX}w=w'g=w'1_Z\\&=w'.
		\end{align*}
		\end{proof}
		%Ej 11
		\item Sean $(\mathscr{C},T,\triangle)$ una categoría triangulada y $\mathscr{D}$ una subcategoría triangulada. Pruebe que:
		\begin{itemize}
			\item[a)] $\mathscr{D}$ es cerrada por isomorfismos en $\mathscr{C}$ (en particular, $\mathscr{D}$ contiene a todos los ceros de $\mathscr{C}$).
			
			\item[b)] $\mathscr{D}$ es una subcategoría aditiva de  $\mathscr{C}$.
			
			\item[c)] $\forall \eta:\xymatrix{&Z\ar[ld]&\\ X\ar[rr] & &Y\ar[ul]}\in \triangle $, con $X,Y$ en $\mathscr{D}$, se tiene que $Z\in \mathscr{D}$.
			
			\item[d)] Si $X\to Y\to Z\to TX \in \triangle$ y dos de los objetos $X,Y,Z$ están en $\mathscr{D}$, entonces el tercero de ellos también lo está.
			
			\item[e)] Sea $\triangle|_\mathscr{D}:=\{X\to Y\to Z\to TX \in \triangle\,:\, X,Y,Z\in \mathscr{D}\}$ y la
			restricción $T|_\mathscr{D}$ del funtor $T:\mathscr{C}\to\mathscr{C}$ en la subcategoría $\mathscr{D}$.\\
			Entonces el triple $(\mathscr{D},T|_\mathscr{D},\triangle|_\mathscr{D})$ es una categoría triangulada.
			
			\item[g)] $\mathscr{D}^{op}$ es una subcategoría triangulada de $(\mathscr{C},T,\triangle)^{op}$.
		\end{itemize}
		
		\begin{proof}
			\boxed{a)} Sean $X,0\in Obj(\mathscr{D})$ tal que $X\cong Y$ en $\mathscr{C}$ (por definición de subcategoría triangulada existe un $0$ en 
			$\mathscr{D}$).\\
			
			Observemos primero que
			
			\begin{equation*}
				\xymatrix@C+2pc{X\ar[r]_{\sim}^{\spmat{1_X \\ 0}}& X\coprod 0}\\
			\end{equation*}
			y que los siguientes son morfismos en $\mathscr{C}$:  $(h\,\,0): X\coprod 0\rightarrow Y$\,\,\,y\,\,\, 
			$\left(\xymatrix@R=-1mm{h^{-1}\\0}\right):Y\rightarrow X\coprod 0$. En particular 
			
			\[(h\,\,0) \left(\xymatrix@R=-1mm{h^{-1}\\0}\right)=hh^{-1} +0=1_Y\qquad \text{y}\]
			\[\left(\xymatrix@R=-1mm{h^{-1}\\0}\right) (h\,\,0)=\left(\xymatrix@R=-1mm{h^{-1}h\\0}\right)
			=\left(\xymatrix@R=-1mm{1_X\\0}\right)=1_{x\coprod 0}\,.\]
			
			Mas aún, si consideramos a $Y$ con la familia $\{\nu_1=h, \nu_2=0_{0Y}\}$ se tiene que $(h\,\,0)$ es un isomorfismo que conmuta con las 
			inclusiones naturales del coproducto $X\coprod 0$ :
			
			\[(h\,\,0) \left(\xymatrix@R=-1mm{1_X\\0}\right)=h+0=h=\nu_1\quad \text{y}\]
			\[(h\,\,0) \left(\xymatrix@R=-1mm{0\\0}\right)=0+0=0=\nu_2.\]
			
			Por lo tanto $Y$ con la familia $\{\nu_1\,,\,\nu_2\}$ son un coproducto de $\{X,0\}$ y como $\mathscr{D}$ es una subcategoría triangulada, entonces 
			por $( ST2 )$ se tiene que $Y\in Obj(\mathscr{D})$.\\
			
			\boxed{b)} Por definición de subcategoría triangulada $\mathscr{D}$ tiene objeto cero \\( SA1 ). Como $\mathscr{C}$ es aditiva y $\mathscr{D}$ es 
			plena, $\operatorname{Hom}_{\mathscr{D}}$ tiene estructura de grupo abeliano para todo $A,B\in Obj(\mathscr{C})$ ( SA2 ).\\
			
			Como $\mathscr{C}$ es aditiva y $\mathscr{D}$ es plena la composición de morfismos en $\mathscr{D}$ es bilineal ( SA3 ), además, por definición
			$\mathscr{D}$ es cerrada bajo coproductos finitos ( SA4 ), asi $\mathscr{D}$ es  subcategoría aditiva de $\mathscr{C}$.\\
			
			\boxed{c)} Sea $\eta: \xymatrix{X\ar[r]^u & Y\ar[r]^v & Z\ar[r]^w & TX } \in \triangle$ con $X,Y\in \mathscr{D}$, entonces por ( TR2 ) 
			$\xymatrix{Y\ar[r]^v & Z\ar[r]^w & TX\ar[r]^{-T(u)} & TY} \in \triangle$, como $T(\mathscr{D})\subseteq \mathscr{D}$, se tiene que $TX\in 
			\mathscr{D}$ y por ( TR1a ) 
			
			\begin{gather*}
				\xymatrix{Y\ar[r]^{1_Y}&Y\ar[r]&0\ar[r]&TY}\quad \text{y}\\
				\xymatrix{0\ar[r]&TX\ar[r]^{T(1_{X})}&TX\ar[r]&0}
			\end{gather*}
			están en $\triangle$.\\
			
			Pero $\mathscr{D}$ es cerrado bajo coproductos finitos, así 
			\begin{equation*}
				\xymatrix@C+2pc{Y\ar[r]^{\spmat{1 \\ 0}} & Y\coprod TX\ar[r]^{\spmat{ 0 & 1}} & TX\ar[r]^{-T(u)} & TY}\quad \in \triangle .
			\end{equation*}
			Así se tiene el siguiente diagrama conmutativo:\\
			
			
			\begin{equation*}
				\xymatrix@C+2pc{Y\ar[r]^{\spmat{1 \\ 0}}\ar[d]^{1_Y} & Y\coprod TX\ar[r]^{\spmat{ 0 & 1}} & TX\ar[d]^{T(1_X)}\ar[r]^{-T(u)} 
					& TY\ar[d]^{T(1_Y)}\\
					Y\ar[r]^v & Z\ar[r]^w & TX\ar[r]^{-T(u)} & T(Y)}
			\end{equation*}
			
			por el lema ( 1.1 ) existe un morfismo $g:Y\coprod TX\longrightarrow Z$ tal que \\$\varphi:=(1_Y,g,T(1_X))\in J(\mathscr{C},\triangle)$, además por
			(Ejercicio 3) se tiene que $g$ es iso, por lo tanto $\varphi$ es iso y por el inciso b) se tiene que $Z\in \mathscr{D}$.\\
			
			\boxed{d)} Sea $X\to Y\to Z\to TX \in \triangle$ tal que dos de los objetos $X,Y,Z$ están en $\mathscr{D}$, si $X,Y\in \mathscr{D}$ entonces el 
			inciso c) implica que $Z\in \mathscr{D}$. Ahora (S.P.G) supongamos $X,Z\in \mathscr{D}$ rotamos a la izquierda por el teorema ( 1.3 ) y tenemos 
			que $T^{-1}(Z)\to X\to Y\to Z\in \triangle$. Pero $T^{-1}(\mathscr{D})\subseteq \mathscr{D}$, por lo que $T^{-1}(Z)\in \mathscr{D}$. 
			Y así, por el inciso c) se tiene el resultado (el último caso es análogo rotando dos veces a la izquierda).\\
			
			\boxed{e)} Sean $\triangle|_{\mathscr{D}}:=\{X\to Y\to Z\to TX\in \triangle\,|\, X,Y,Z\in \mathscr{D}\}$ y la restricción $T|_{\mathscr{D}}$ del
			funtor $T:\mathscr{C}\longrightarrow \mathscr{C}$ en la subcategoría $\mathscr{D}$.\\
			
			Se probará que $(\mathscr{D},T|_{\mathscr{D}},\triangle|_{\mathscr{D}})$ es una categoría triangulada (probando cada uno de los axiomas).\\
			
			\boxed{TR1(a)}\\
			Sea $X\in \mathscr{D}$, en particular  $X\in \mathscr{C}$ por lo que, por ( TR1.a ) sobre $\mathscr{C}$
			\\$\xymatrix{X\ar[r]^{1_X}&X\ar[r]&0\ar[r]&TX\quad\in \triangle}$.
			Pero por el inciso a) sabemos que todos los ceros de $\mathscr{C}$ están en $\mathscr{D}$, y como $TX\in \mathscr{D}$  entonces \\
			$\xymatrix{X\ar[r]^{1_X}&X\ar[r]&0\ar[r]&TX\quad\in \triangle|_{\mathscr{D}}}$\\
			
			\boxed{TR1(b)}\\
			Por el inciso a) se tiene el resultado.\\
			
			\boxed{TR1(c)}\\
			Sea $f:X\to Y$ en $\mathscr{D}$, entonces $f:X\to Y\in \mathscr{C}$ y por ( TR1.c ) en $\mathscr{C}$ existe
			$\xymatrix{Y\ar[r]^{\alpha}&Z\ar[r]^{\beta}&TX}\in \mathscr{C}$ tal que \\
			$\xymatrix{X\ar[r]^{f}&Y\ar[r]^{\alpha}&Z\ar[r]^{\beta}&TX}\in \triangle$. Por el inciso c), como $X,Y\in \mathscr{D}$ entonces $Z\in \mathscr{D}$
			y como $\mathscr{D}$ es plena se tiene que \\$\xymatrix{X\ar[r]^{f}&Y\ar[r]^{\alpha}&Z\ar[r]^{\beta}&TX}\in \triangle|_{\mathscr{D}}$.\\
			
			\boxed{TR2}\\
			Supongamos $\xymatrix{X\ar[r]^{u}&Y\ar[r]^{v}&Z\ar[r]^{w}&TX}\in \triangle|_{\mathscr{D}}\subseteq \triangle$ entonces \\
			$\xymatrix{Y\ar[r]^{v}&Z\ar[r]^{w}&TX\ar[r]^{-T(u)}&TY}\in \triangle$ pero $T(\mathscr{D})\subset \mathscr{D}$ y $T$ es pleno por lo que 
			$\xymatrix{Y\ar[r]^{v}&Z\ar[r]^{w}&TX\ar[r]^{-T(u)}&TY}\in \triangle|_{\mathscr{D}}$.\\
			
			\boxed{TR3}\\
			Sean $\eta=(X,Y,Z,u,v,w)$, $\mu=(X',Y',Z',u',v',w')$ en $\triangle|_{\mathscr{D}}$ y\\$f:X\to X',\,\, g:Y\to Y'$ en $\mathscr{D}$ tales que $gu=u'f$.\\
			Por ( TR3 ) sobre $\mathscr{C}$ existe $h:Z\to Z'$ tal que $\varphi:=(f,g,h):\eta\to \mu$ es morfismo en $\mathscr{T}(\mathscr{C},T)$ en particular
			como $Z,Z'\in \mathscr{D}$ y $\mathscr{D}$ es plena, entonces $h\in \hom_{\mathscr{D}}(Z,Z')$, por lo tanto $\varphi\in \mathscr{T}(\mathscr{D},T)$.
			\\
			
			\boxed{TR4} (Axioma del octaedro)\\
			Supongamos que tenemos el siguiente diagrama en $\mathscr{D}$:\\
			\centerline{
				\xymatrix{
					X\ar@^{=}[d]\ar[r]^u&Y\ar[d]^v\ar[r]^i&Z'\ar[r]^{i'}&TX\ar@^{=}[d] &\in \triangle|_{\mathscr{D}}\\
					X\ar[r]^{vu}\ar[d]^u&Z\ar[r]^k\ar@^{=}[d]&Y'\ar[r]^{k'}&TX\ar[d]^{T(u)} &\in \triangle|_{\mathscr{D}}\\
					Y\ar[r]^{v}&Z\ar[r]^{r}&X'\ar[r]^{j}\ar[d]^{T(i)j}&TY\ar[d]^{T(i)} &\in \triangle|_{\mathscr{D}}\\
					&&TZ'\ar@^{=}[r]&TZ'\,.
				}
			} 
			El axioma del octaedro en $\mathscr{C}$ nos indica que existe $f:Z'\to Y'$ y $g:Y'\to X'$ \,tales que hacen conmutar el diagrama anterior, y que\\
			$\xymatrix{\theta:Z'\ar[r]^f&Y'\ar[r]^g&Z'\ar[r]^{h\quad}&TZ' \,\,\, \in \triangle}$. Como $\mathscr{D}$ es pleno y cada objeto (vertice) 
			del diagrama está en 
			$\mathscr{D}$ entonces $f,g\in Mor(\mathscr{D})$ por lo que $\theta\in  \triangle|_{\mathscr{D}}$ y se cumple el axioma del octaedro.\\
			
			\boxed{f)} Por alguna extraña razón no hay f en este ejercicio.\\
			
			\boxed{g)} Para comenzar se observa que, como $\mathscr{D}$ es una categoría triangulada, entonces por el ( Ej. 5 )  $\mathscr{D}^{op}$ 
			será una categoría triangulada. También se tiene que $Obj(\mathscr{D})=Obj(\mathscr{D}^{op})$ y como $\mathscr{D}$ es subcategoría plena de 
			$\mathscr{C}$ entonces $\mathscr{D}^{op}$ es subcategoría plena de $\mathscr{C}^{op}$. Esto último es facil de ver, pues cada morfismo en 
			$\mathscr{C}^{op}$ entre objetos de $\mathscr{D}^{op}$ es el morfismo opuesto $f^{op}$ de un morfismo $f$ en $\mathscr{C}$ entre objetos de
			$\mathscr{D}$, y como $\mathscr{D}$ es plena, entonces $f$ está en $\mathscr{D}$ y así $f^{op}$ está en $\mathscr{D}^{op}$.\\
			
			Se probarán los axiomas de subcategoría triangulada para $(\mathscr{D},T|_\mathscr{D},\triangle|_\mathscr{D})$.\\
			
			\boxed{ST1} Como $\mathscr{D}$ contiene un objeto cero, entonces $\mathscr{D}^{op}$ contiene un objeto cero.\\
			
			\boxed{ST2} Sean $X,Y\in\mathscr{D}^{op}$, entonces $X,Y\in\mathscr{D}$ y por ser $\mathscr{D}$ subcategoría triangulada, entonces $X\coprod Y$ 
			está en $\mathscr{D}$. Pero $\mathscr{C}$ es una categoría aditiva, entonces $\mathscr{D}$ es aditiva también y todo coproducto finito es un producto 
			finito, así $X\prod Y\in \mathscr{D}$ y por lo tanto $\left(X\coprod^{op} Y\right)=\left(X\prod Y\right)^{op}\in \mathscr{D}^{op}$ donde 
			$\coprod^{op}$ denota al coproducto en $\mathscr{D}$.\\
			
			\boxed{ST3} Observemos que $ A\in Obj(\mathscr{D}^{op})\,\,\iff\,\, A\in Obj(\mathscr{D})$, entonces para cada $A\in \mathscr{D}, \,\,
			T(A)\in \mathscr{D}$ y $\tilde{T}(A)\in \mathscr{D}^{op}$. Análogamente $T^{-1}(A)\in \mathscr{D}^{op}$.\\
			
			Sea $f:A\to B$ en $\mathscr{D}^{op}$ entonces $\tilde{T}(f)=(T^{-1}(f^{op}))^{op}$ pero $f^{op}:B\to A$ está en $\mathscr{D}$ por lo que
			$T^{-1}(f^{op})\in \mathscr{D}$, es decir, $\tilde{T}(f)\in (\mathscr{D})^{op}$. Análogamente $\tilde{T}^{-1}(f)\in (\mathscr{D})^{op}$.\\
			
			\boxed{ST4} Sea $f:X\to Y$ en $\mathscr{D}^{op}$, ntonces $f^{op}:Y\to X$ en $\mathscr{D}$, por ( ST4 ) sobre $\mathscr{D}$, se tiene que
			existe $\eta:\xymatrix{Y\ar[r]^{f^{op}}&X\ar[r]&Z\ar[r]&TY\,\,\in \triangle|_{\mathscr{D}}}$ con $Z$ en $\mathscr{D}$. Así por el teorema ( 1.3 ) 
			$\eta_0:\xymatrix{T^{-1}Z\ar[r]&Y\ar[r]^{f^{op}}&X\ar[r]&Z\,\,\in \triangle|_{\mathscr{D}}}$, y por definición de categoría triangulada opuesta\\
			$\xymatrix{X\ar[r]^f&Y\ar[r]&\tilde{T}^{-1}Z\ar[r]&\tilde{T}X\,\,\in \triangle|_{\mathscr{D}^{op}}}$.\\
			
			Por lo que $(\mathscr{D},T|_\mathscr{D},\triangle|_\mathscr{D})^{op}$ es subcategoría triangulada de $(\mathscr{C},T,\triangle)^{op}$.
			
		\end{proof}
		%Ej 12
		\item Sea $(F,\eta):(\mathscr{C}_{1},T_{1},\triangle_{1})\to (\mathscr{C}_{2},T_{2},\triangle_{2})$ un funtor graduado entre categor\'ias trianguladas. Para cada $\theta : \xymatrix{X\ar[r]^u & Y\ar[r]^v & Z\ar[r]^w & TX } \in T(\mathscr{C}_{1},T_{1})$ defina $\bar{F}(\theta):\xymatrix{FX\ar[r]^Fu & FY\ar[r]^Fv & FZ\ar[r]^Fw & T_{2}FX } \in T(\mathscr{C}_{2},T_{2})$. 
		
		\bigskip
		
		Pruebe que la correspondencia que extiende al funtor $F$ a la categor\'ia de tri\'angulos $\bar{F} :\mathcal{T}(\mathscr{C}_{1},T_{1})\to \mathcal{T}(\mathscr{C}_{2},T_{2})$
		\begin{center}
			$((f,g,h):\theta \to \mu)\longmapsto (\bar{F}(f,g,h):\bar{F}(\theta)\to \bar{F}(\mu))$ 
		\end{center} 
		donde $\bar{F}(f,g,h)=(Ff,Fg,Fh)$, es funtorial.
		
		
		\begin{proof}
			\begin{enumerate}
				\item Sea $\varphi=(f,g,h):\theta \to \mu$ un morfismo en $\mathcal{T}(\cc_{1},T_{1})$, se demuestra que en efecto $\bar{F}(\varphi):\bar{F}(\theta) \to \bar{F}(\mu)$ es un morfismo en $\mathcal{T}(\cc_{2},T_{2})$. A continuación se ilustra al morfismo $\varphi:\theta \to \mu$:
				
				\begin{center}
					\begin{tikzcd}
						X \arrow{r}{u} \arrow{d}{f} & Y \arrow{r}{v} \arrow{d}{g} & Z \arrow{d}{h} \arrow{r}{w} & T_{1}X \arrow{d}{T_{1}f} \\
						A \arrow{r}{a} & B \arrow{r}{b} & C \arrow{r}{c} & T_{1}A
					\end{tikzcd}
				\end{center}
				
				se busca demostrar que el siguiente diagrama conmuta:
				
				\begin{center}
					\begin{tikzcd}
						FX \arrow{r}{Fu} \arrow{d}{Ff} & FY \arrow{r}{Fv} \arrow{d}{Fg} & FZ \arrow{d}{Fh} \arrow{r}{\eta_{X}\circ Fw} & T_{2}FX \arrow{d}{T_{1}f} \\
						FA \arrow{r}{Fa} & FB \arrow{r}{Fb} & FC \arrow{r}{\eta_{A} \circ Fc} & T_{2}FA
					\end{tikzcd}
				\end{center}
				
				observe que debido a la funtorialidad de $F$ se tiene la conmutatividad de los primeros dos cuadros, resta verificar la del \'ultimo.
				
				\bigskip
				
				Como $\eta:FT_{1}\to T_{2}F$ es un isomorfismo natural, se tiene que el siguiente diagrama conmuta:
				
				\begin{center}
					\begin{tikzcd}
						FZ \arrow{d}{Fh} \arrow{r}{Fw} & FT_{1}X \arrow{d}{FT_{1}f} \arrow{r}{\eta_{X}} & T_{2}FX \arrow{d}{T_{2}Ff}\\
						FC \arrow{r}{Fc} & FT_{1}A \arrow{r}{\eta_{A}} & T_{2}FA
					\end{tikzcd}
				\end{center}
				
				as\'i pues se puede concluir que $\bar{F}(\varphi):\bar{F}(\theta)\to \bar{F}(\mu)$ es un morfismo en $\mathcal{T}(\cc_{2},T_{2})$.
				
				\bigskip
				
				En lo que respecta a la composici\'on se cumple lo siguiente. Sean $\varphi=(f,g,h):\theta \to \mu$, $\psi=(r,s,t):\mu \to \sigma$ morfismos en $\mathcal{T}(\cc_{1},T_{1})$. As\'i
				
				\begin{align*}
					\bar{F}(\psi \circ \varphi) =& \bar{F}((r,s,t)\circ (f,g,h))\\
					=& \bar{F}(rf,sg,th)\\
					=& (F(rf),F(sg),F(th))\\
					=& (Fr\circ Ff,Fs\circ Fg,Ft\circ Fh)\\
					=& (Fr,Fs,Ft)\circ (Ff,Fg,Fh)\\
					=& \bar{F}(r,s,t)\circ \bar{F}(f,g,h)
				\end{align*}
				
				Adem\'as para cualquier $\theta : \xymatrix{X\ar[r]^u & Y\ar[r]^v & Z\ar[r]^w & TX } \in \mathcal{T}(\mathscr{C}_{1},T_{1})$ se cumple que:
				
				\begin{align*}
					\bar{F}(1_{\theta}) =& \bar{F}(1_{X},1_{Y},1_{Z})\\
					=& ( F(1_{X}),F(1_{Y}),F(1_{Z}) )\\
					=& (1_{FX},1_{FY},1_{FZ})\\
					=& 1_{\bar{F}(\theta)}.
				\end{align*}
				
				Se puede concluir que $\bar{F}:\mathcal{T}(\mathscr{C}_{1},T_{1})\to \mathcal{T}(\mathscr{C}_{2},T_{2})$ es un funtor.
				
			\end{enumerate}
		\end{proof}
		%Ej 13
		\item Sea $\catarrow{\eta}{\lrprth{F,\alpha}}{\lrprth{G,\beta}}{}$ una transformación de funtores graduados con $F, G\in\lrsqp{\mathscr{C}_1,\mathscr{C}_1}$, $\overline{F}$ y $\overline{G}$ los funtores introducidos en el Ej. 12 y, para cada
		\begin{equation*}
			\epsilon:\shortseq{lcr=t,f=u,g=v,h=w,T=T_1,}\in\mathscr{T}\lrprth{\mathscr{C}_1,T_1},
		\end{equation*} 
		$\overline{\eta}_\epsilon:=\lrprth{\eta_X,\eta_Y,\eta_Z}$. Entonces $\overline{\eta}:=\lrbrack{\overline{\eta}_\epsilon}_{\epsilon\in\mathscr{T}\lrprth{\mathscr{C}_1,T_1}}\in\nattrans{\mathscr{C}_1}{\mathscr{C}_2}{\overline{F}}{\overline{G}}$.
		\begin{proof}
			Afirmamos que $\forall\ \epsilon$
			\begin{equation*}
				\tmorph{A=FX,B=FY,C=FZ,f=Fu,g=Fv,h=\alpha_XFw,T=T_2,Ap=GX,Bp=GY,Cp=GZ,fp=Gu,gp=Gv,hp=\beta_XGw,p=\eta_X,q=\eta_Y,r=\eta_Z,I=\text{(I)},I=\text{(II)},I=\text{(III)}}
			\end{equation*}			
			es un diagrama conmutativo en $\mathscr{C}$. En efecto, la conmutatividad de (I) y (II) se siguen de que $\catarrow{\eta}{F}{G}{}$ es una transformación natural.\\
			Ahora, dado que $\eta$ es una transformación natural, para el morfismo $\catarrow{w}{Z}{T_1X}{}$ se tiene que
			\begin{equation*}
				\commutativesquare{A=FZ,B=GZ,C=FT_1X,D=GT_1X,f=\eta_Z,g=Fw,h=Gw,k=\eta_{T_1X},}
			\end{equation*}
			 es un diagrama conmutativo en $\mathscr{C}$,  al igual que lo es el siguiente dado que $\eta$ es una transformación natural de funtores graduados
			 \begin{equation*}
			 	\commutativesquare{A=FT_1X,B=T_2FX,C=GT_1X,D=T_2GX,f=\alpha_X,g=\eta_{T_1X},h=T_2\eta_X,k=\beta_X,},
			 \end{equation*}
		 	de lo cual se sigue que
		 	\begin{align*}
		 		\lrprth{T_2\eta_X}\alpha_XFw&=\lrprth{\lrprth{T_2\eta_X}\alpha_X}Fw=\lrprth{\beta_X\eta_{T_1X}}Fw\\
		 		&=\beta_X\lrprth{\eta_{T_1X}Fw}=\beta_X\lrprth{Gw\eta_Z},\\
		 		&=\lrprth{\beta_XGw}\eta_Z.
		 	\end{align*}
	 		Lo anterior garantiza la conmutatividad de (III), con lo cual se ha verificado la afirmación. Por lo tanto $\forall\ {\epsilon\in\mathscr{T}\lrprth{\mathscr{C}_1,T_1}}$ se tiene que $\overline{\eta}_\epsilon=\lrprth{\eta_X,\eta_Y,\eta_Z}\in\ringmodhom{\mathscr{T}\lrprth{\mathscr{C}_2,T_2}}{\overline{F}}{\overline{G}}$.\\
	 		
	 		Ahora, sea $\catarrow{\varphi=\lrprth{f,g,h}}{\delta}{\delta'}{}$ en $\mathscr{T}\lrprth{\mathscr{C}_1,T_1}$, con 
	 		\begin{align*}
	 			\delta&:\shortseq{lcr=t,A=A,B=B,C=C,T=T_1,f=r,g=s,h=t,},\\
	 			\delta'&:\shortseq{lcr=t,A=A',B=B',C=C',T=T_1,f=r',g=s',h=t',}.
	 		\end{align*}
 			Luego, por ser $\catarrow{\eta}{F}{G}{}$ una transformación natural de los morfismos $f,g$ y $h$ se obtienen los siguientes diagramas conmutativos en $\mathscr{C}$
 			\begin{align*}
 				\commutativesquare{A=FA,B=GA,C=FA',D=GA',f=\eta_A,g=Ff,h=Gf,k=\eta_{A'},}&,\\
 				\commutativesquare{A=FB,B=GB,C=FB',D=GB',f=\eta_B,g=Fg,h=Gg,k=\eta_{B'},}&,\\
 				\commutativesquare{A=FC,B=GC,C=FC',D=GC',f=\eta_C,g=Fh,h=Gh,k=\eta_{B'},}&,
 			\end{align*}
 			con lo cual
	 		\begin{align*}
	 			\overline{G}\varphi\eta_\delta&=\lrprth{Gf,Gg,Gh}\circ\lrprth{\eta_A,\eta_B,\eta_C}=\lrprth{Gf\eta_A,Gg\eta_B,Gh\eta_C}\\
	 			&=\lrprth{\eta_{A'}Ff,\eta_{B'}Fg,\eta_{C'}Fh}=\lrprth{\eta_{A'},\eta_{B'},\eta_{C'}}\circ\lrprth{Ff,Fg,Fh}\\
	 			&=\eta_{\delta'}\overline{F}\varphi.
	 		\end{align*}
	 		Así
	 		\begin{equation*}
	 			\commutativesquare{A=\overline{F}\delta,B=\overline{G}\delta,C=\overline{F}\delta',D=\overline{G}\delta',f=\eta_\delta,g=\overline{F}\varphi,h=\overline{G}\varphi,k=\eta_{\delta'},}
	 		\end{equation*}
 		es un diagrama conmutativo en $\mathscr{C}$.\\
 		
 		Por todo lo anterior $\overline{\eta}$ es una transformación natural de $\overline{F}$ en $\overline{G}$.\\
		\end{proof}
	\end{enumerate}		
	\section*{Ejercicios no numerados}
	\begin{propsn}[\textbf{Lema 1.1(b)}]
		Sean $\lrprth{\mathscr{C},T,\Delta}$ una categoría pre-triangulada y $\eta=\lrprth{X,Y,Z,u,v,w}$, $\eta'=\lrprth{X',Y',Z',u',v',w'}\in\Delta$. Si $\exists\ \catarrow{g}{Y}{Y'}{}$ y $\exists\ \catarrow{h}{Z}{Z'}{}$ tales que $hv=v'g$, entonces $\exists\ \catarrow{f}{X}{X'}{}$ tal que $\lrprth{f,g,h}$ es un morfismo de $\eta$ en $\eta'$.
	\end{propsn}
	\begin{proof}
		Como $\eta,\eta'\in\Delta$, entonces por TR2 los siguientes son triángulos dustinguidos
		\begin{align*}
			\mu&:\shortseq{lcr=t,A=Y,B=Z,C=TX,f=v,g=w,h=-Tu,},\\
			\mu'&:\shortseq{lcr=t,A=Y',B=Z',C=TX',f=v',g=w',h=-Tu',}
		\end{align*}
		para los cuales, por hipótesis, $\exists\ g\in\ringmodhom{\mathscr{C}}{Y}{Y'}$ y $\exists\ h\in\ringmodhom{\mathscr{C}}{Z}{Z'}$ tales que
		\begin{center}
			\begin{tikzcd}
				Y\arrow{rr}{v}\arrow{dd}[swap]{g} & & Z
				\arrow{rr}{w}\arrow{dd}{h} & &	TX\arrow{rr}{-Tu}\ar[dd,dashed, "\exists\ \Psi"]& & TY\ar[dd,"Tg"]\\
				& \text{(I)}& & \text{(II)} & & \text{(III)}&\\
				Y'\arrow{rr}{v'} & &Z'\arrow{rr}{w'} & & TX'\arrow{rr}{-Tu'}& & TY
			\end{tikzcd}
		\end{center}
		(I) y (II) son diagramas conmutativos en $\mathscr{C}$. Luego por $TR3$ $\exists\Psi\in\ringmodhom{\mathscr{C}}{TX}{TX'}$ tal que $\lrprth{g,h,\psi}\in\ringmodhom{\mathscr{T}\lrprth{\mathscr{C},T}}{\mu}{\mu'}$. Dado que $T$ es en partícular pleno, $\exists\ f\in\ringmodhom{\mathscr{C}}{X}{X'}$ tal que $Tf=\Psi$.\\
		Como se tiene ahora que (III) es un diagrama conmutativo en $\mathscr{C}$, se tiene que
		\begin{align*}
			-T\lrprth{gu}&=\lrprth{Tg}\lrprth{-Tu}=\lrprth{-Tu'}\Psi\\
			&=\lrprth{-Tu'}Tf=-T\lrprth{u'f},
		\end{align*} 
		y así $gu=u'f$, pues $T$ es, en partícular, fiel. Por su parte, de (II) se sigue que $\lrprth{Tf}w=w'h$, de modo que
		\begin{equation*}
			\tmorph{Ap=X',Bp=Y',Cp=Z',f=u',g=v',h=w',p=f,q=g,r=h,}
		\end{equation*}
		 es un diagrama conmutativo en $\mathscr{C}$ y por lo tanto $\lrprth{f,g,h}\in\ringmodhom{\mathscr{T}\lrprth{\mathscr{C},T}}{\eta}{\eta'}$.\\
	\end{proof}

	\begin{propsn}[Lema 1.12]
		Sean $\lrprth{\mathscr{C},T,\Delta}$ una categoría triangulada y $\mathscr{D}\subseteq\mathscr{C}$, entonces:
		\begin{enumerate}[label=(\roman*)]
			\item $\mathscr{D}$ es una subcategoría triangulada de $\lrprth{\mathscr{C},T,\Delta}$ si y sólo si $T\lrprth{\mathscr{D}}\subseteq\mathscr{D}$ y $\mathscr{D}$ es cerrada por co-conos;
			\item $\mathscr{D}$ es una subcategoría triangulada de $\lrprth{\mathscr{C},T,\Delta}$ si y sólo si $T\lrprth{\mathscr{D}}\subseteq\mathscr{D}$, $T^{-1}\lrprth{\mathscr{D}}\subseteq\mathscr{D}$ y $\mathscr{D}$ es cerrada por extensiones.
		\end{enumerate}
		\begin{proof}
			\boxed{(\text{i})\implies} De $ST3$ se sigue que $T^{-1}\lrprth{D}\subseteq{D}$.\\
			Ahora, sea
			\begin{equation*}
				\shortseq{lcr=t,f=u,g=v,h=w,}\in\Delta.
			\end{equation*}
			Por Prop. 1.3 
			\begin{equation*}
				\shortseq{lcr=m,A=T^{-1}Z,B=X,C=Y,T=Z,f=-Tw,g=u,h=v,}\in\Delta,
			\end{equation*}
			con $X,Y\in \mathscr{D}$, luego por Ej. 11(d) $T^{-1}Z\in\mathscr{D}$. Por lo tanto $\mathscr{D}$ es cerrado por co-conos.\\
			
			\boxed{(\text{i})\impliedby} \underline{ST1}. Sea $X\in\mathscr{D}$ y $0$ un objeto cero en $\mathscr{C}$. Por TR1(a) 
			\begin{equation*}
				\shortseq{lcr=t,B=Y,C=0,f=1,}\in\Delta,
			\end{equation*}
		luego $T^{0}\in\mathscr{D}$, con $T^{-1}0$ un objeto cero de $\mathscr{C}$ puesto que $0$ lo es y $T$ es un automorfismo.\\
		\underline{ST2}. Sean $X,Y\in\mathscr{D}$. Por Coro 1.6 se tiene que
		\begin{align*}
			&\shortseq{lcr=m,A=TX,B=TX\coprod TY,C=TY,T=T^2X,f=\begin{pmatrix*}[c]
					1_{TX}\\
					0
				\end{pmatrix*},g=\begin{pmatrix*}[c]
				0 & 1_{TY}
			\end{pmatrix*}, h=0,}\in\Delta,\\
		\implies &\shortseq{lcr=t, A=Y,B=TX,C=TX\coprod TY,f=-T^{-1}0,g=\begin{pmatrix*}[c]
				1_{TX}\\
				0
			\end{pmatrix*},h=\begin{pmatrix*}[c]
				0 & 1_{TY}
			\end{pmatrix*},}\in \Delta, && TR2\\
		\implies &\shortseq{lcr=t, A=Y,B=TX,C=T\lrprth{X\coprod Y},f=0,g=\begin{pmatrix*}[c]
				1_{TX}\\
				0
			\end{pmatrix*},h=\begin{pmatrix*}[c]
				0 & 1_{TY}
			\end{pmatrix*},}\in \Delta, && TR2\\
		\intertext{esto último puesto que $T$ es un automorfismo. Dado que $\mathscr{D}$ es cerrado por co-conos se sigue que}
		X\coprod Y&= T^{-1}\lrprth{T\lrprth{X\coprod Y}}\in\mathscr{D}.
		\end{align*}
		\underline{ST3}. Resta probar que $T^{-1}\lrprth{\mathscr{D}}\subseteq\mathscr{D}$. Sea $X\in\mathscr{D}$. Por TR1($a$) se tiene que
		\begin{equation*}
			\shortseq{lcr=m,A=T^{-1}X,B=T^{-1}X,C=0,T=X,f=1,}\in\mathscr{D},
		\end{equation*}
		así que por TR2
		\begin{equation*}
			\shortseq{lcr=m,A=T^{-1}X,B=0,C=X,T=X,h=1,}\in\mathscr{D},
		\end{equation*}
		de modo que, por ser $\mathscr{D}$ cerrada por co-conos, $T^{-1}X\in\mathscr{D}$. Por lo tanto $T^{-1}\lrprth{\mathscr{D}}\subseteq\mathscr{D}$.\\
		\underline{ST4}. Sea $\catarrow{f}{X}{Y}{}$ en $\mathscr{C}$ con $X,Y\in\mathscr{D}$. Así
		\begin{align*}
			\exists&\shortseq{lcr=t,f=f,g=g,h=h,}\in\Delta, && TR1(c)\\
			\implies& T^{-1}Z\in\mathscr{D}, && \mathscr{D}\text{ es cerrada por co-conos}\\
			\implies & Z=T\lrprth{T^{-1}Z}\in\mathscr{D}, && T\lrprth{\mathscr{D}}\subseteq\mathscr{D}
		\end{align*}
		con lo cual se verifica el axioma ST4.\\
		Por todo lo anterior se sigue que $\mathscr{D}$ es una subcategoría triangulada de $\mathscr{C}$.\\
		
		\boxed{(\text{ii})\implies} Se sigue de ST3 y Ej. 11(d).\\
		\boxed{(\text{ii})\impliedby} \underline{ST1} Sea $X\in \mathscr{D}$. Así
		\begin{align*}
			\phantom{a}&\shortseq{lcr=t,B=X,C=0,f=1,}\in\Delta, && TR1(a)\\
			&\implies\shortseq{lcr=t,B=0,C=TX,h=1,}\in\Delta, && TR2\\
			&\implies 0\in\mathscr{D}. &&\parbox[t]{1in}{\raggedright $\mathscr{D}$ es cerrada por extensiones}
		\end{align*}		
		\underline{ST2} Se tiene por ser $\mathscr{D}$ cerrado por extensiones y Coro 1.6.\\
		\underline{ST3} Se tiene por hipótesis.\\
		\underline{ST4} Sea $\catarrow{f}{X}{Y}{}$ en $\mathscr{C}$ con $X,Y\in\mathscr{D}$. Así
		\begin{align*}
			\exists&\shortseq{lcr=t,f=f,g=g,h=h,}\in\Delta, && TR1(c)\\
			\implies& \shortseq{lcr=t,A=Y,B=Z,C=TX,f=g,g=h,h=-Tf,}\in\Delta, && TR2\\
			\implies & Z\in\mathscr{D}. && \mathscr{D}\text{ es cerrada por co-conos}
		\end{align*}
		\end{proof} 
	\end{propsn}
	
	\begin{propsn}[\textbf{Corolario 1.9}]
		Para una categoría pretriangulada $(\mathscr{C},T,\triangle)$, las siguientes condiciones se satisfacen:
	\begin{itemize}
		\item[a)] Tomo mono (respectivamente epi) es $\mathscr{C}$ es split-mono (respectivamente split-epi).
		\item[b)] Si $\mathscr{C}$ es abeliano, entonces $\mathscr{C}$ es semisimple ( i.e. \\ $Ext^{1}_\mathscr{C}(X,Y)=0\quad \forall X,Y\in \mathscr{C}\,)$.
	\end{itemize}
	\end{propsn}
	\begin{proof}
		Se mostrará que todo epi en $\mathscr{C}$ es un split-epi.\\
		
		Sea $f:X\to Y$ epi en $\mathscr{C}$, entonces $f^{op}:Y\to X$ es un mono en $\mathscr{C}^{op}$. Por el ( ej. 5 ) sabemos que 
		$(\mathscr{C}^{op},\tilde{T},\tilde{\triangle})$ es una categoría pretriangulada, y por el inciso a), como $f^{op}$es mono, entonces $f^{op}$ es 
		split-mono. Así $f$ es split-epi en $\mathscr{C}^{op}$.\\
	\end{proof}
	
	\begin{propsn}[\textbf{Prop. 1.11(b)}]
		Sea $(\cc,T,\triangle)$ una categor\'ia triangulada. Para cualquier 
		$\xymatrix{A\ar[r]^f & B\ar[r]^g & C\ar[r]^h & TA }\in \triangle$ y $\beta:C\to C$ en $\cc$, se tiene el siguiente diagrama conmutativo en $\cc$:
	\begin{center}
		\begin{tikzcd}
			& W \arrow{d}{} \arrow{r}{1_{W}} & W \arrow{d}{} & \\
			A \arrow{r}{} \arrow{d}{1_{A}} & E \arrow{r}{} \arrow{d}{} & D \arrow{d}{\beta} \arrow{r}{} & TA \arrow{d}{1_{TA}} \\ 
			A \arrow{r}{f} & B \arrow{d}{} \arrow{r}{g} & C \arrow{d}{} \arrow{r}{h} & TA \\
			& TW \arrow{r}{1_{TW}} & TW & \\
		\end{tikzcd}
	\end{center}
	donde las filas y columnas, de dicho diagrama, son triángulos distinguidos.
	\end{propsn}
	
	\begin{proof}
		Por hipótesis se tiene la siguiente configuración inicial:
		
		\begin{center}
			\begin{tikzcd}
				A \arrow{r}{f} & B \arrow{r}{g} & C \arrow{r}{h} & TA \in \triangle \\
				& & D \arrow{u}{\beta} & 
			\end{tikzcd}
		\end{center}
		
		pasando a la categoría opuesta se tiene la siguiente situación:
		
		\begin{center}
			\begin{tikzcd}
				C \arrow{d}{\beta^{op}} \arrow{r}{g^{op}} & B \arrow{r}{f^{op}} & A \arrow{r}{\bar{T}(h^{op})} & \bar{T}C \in \overline{\triangle}\\
				D & & & 
			\end{tikzcd}
		\end{center}
		
		ahora, aplicando el \emph{cambio de cobase} se tiene el siguiente diagrama conmutativo en $\cc^{op}$ donde filas y columnas son triángulos en $\overline{\triangle}$:
		
		\begin{center}
			\begin{tikzcd}
				& \overline{T}^{-1}W \arrow{d}{p^{op}} \arrow{r}{1_{\overline{T}^{-1}W}} & \overline{T}^{-1}W \arrow{d}{r^{op}} & \\
				\overline{T}^{-1}A \arrow{r}{-\overline{T}^{-1}(\overline{T}(h^{op}))} \arrow{d}{1_{A}} & C \arrow{r}{g^{op}} \arrow{d}{\beta^{op}} & B \arrow{d}{s^{op}} \arrow{r}{f^{op}} & A \arrow{d}{1_{A}} \\ 
				\overline{T}^{-1}A \arrow{r}{a^{op}} & D \arrow{d}{q^{op}} \arrow{r}{b^{op}} & E \arrow{d}{t^{op}} \arrow{r}{c^{op}} & A \\
				& W \arrow{r}{1_{W}} & W & \\
			\end{tikzcd}
		\end{center}
		
		si rotamos a una vez a la derecha (por TR2) a los triángulos distinguidos que ocupan las filas centrales se obtiene la siguiente situaci\'on:
		
		\begin{center}
			\begin{tikzcd}
				\overline{T}^{-1}W \arrow{d}{p^{op}} \arrow{r}{1_{\overline{T}^{-1}W}} & \overline{T}^{-1}W \arrow{d}{r^{op}} & & \\
				C \arrow{r}{g^{op}} \arrow{d}{\beta^{op}} & B \arrow{d}{s^{op}} \arrow{r}{f^{op}} & A \arrow{d}{1_{A}} \arrow{r}{-\overline{T}(-h^{op})} & \overline{T}C \arrow{d}{\overline{T}(\beta^{op})} \\ 
				D \arrow{d}{q^{op}} \arrow{r}{b^{op}} & E \arrow{d}{t^{op}} \arrow{r}{c^{op}} & A \arrow{r}{-\overline{T}(a^{op})} & \overline{T}D \\
				W \arrow{r}{1_{W}} & W & & \\
			\end{tikzcd}
		\end{center}
		
		dado que $b^{op}\circ -h^{op}=a^{op}$ se deduce que $\overline{T}(\beta^{op})\circ -\overline{T}(-h^{op})=-\overline{T}(a^{op})$. As\'i pues el diagrama anterior es conmutativo.
		
		\bigskip
		
		De modo que pasando a la categoría opuesta los triángulos que ocupan las filas centrales y reescribiendo algunos morfismos se tiene el siguiente diagrama conmutativo en $\cc$:
		
		\begin{center}
			\begin{tikzcd}
				& TW  \arrow{r}{1_{TW}} & TW & \\
				A \arrow{r}{f} & B \arrow{r}{g} \arrow{u}{r} & C \arrow{u}{p} \arrow{r}{h} & TA \\ 
				A \arrow{r}{c} \arrow{u}{1_{A}} & E \arrow{u}{s} \arrow{r}{b} & D \arrow{u}{\beta} \arrow{r}{a} & TA \arrow{u}{1_{A}} \\
				& W \arrow{u}{} \arrow{r}{1_{W}} & W \arrow{u}{q} & \\
			\end{tikzcd}
		\end{center}
		
		observe que cada fila y columna es un triangulo en $\triangle$ como se buscaba. Se concluye el ejercicio.\\
	\end{proof}

	\begin{propsn}[\textbf{Ej. 13'}]
		Sean $\mathscr{C}$ una categoría y $\Sigma\subset Mor(\mathscr{C})$. Pruebe que la categoría $\mathscr{C}[\Sigma^{-1}]$ y el 
		funtor de localización $Q:\mathscr{C}\rightarrow \mathscr{C}(\Sigma^{-1}]$ son únicos salvo isomorfismos. Mas precisamente, sea $q:\mathscr{C}
		\rightarrow \mathscr{B}$ un funtor tal que 
		\begin{itemize}
			\item[a)] $\forall \sigma\in \Sigma$ \quad $q(\sigma)$ es iso.
			\item[b)] $\forall f:\mathscr{C}\to \mathscr{A}$ tal que $F(\sigma)$ es iso $\forall \sigma\in \Sigma$, $\exists ! \bar{F}:\mathscr{B}\to\mathscr{A}$
			tal que $\bar{F}\circ q=F$.\\
		\end{itemize}
		Pruebe que existe un isomorfismo de categorías $\epsilon:\mathscr{C}[\Sigma^{-1}]\to \mathscr{B}$ tal que \\$\epsilon\circ Q=q$ \quad i.e. \\
		\centerline{
			\xymatrix{
				\mathscr{C}\ar[dr]_q\ar[r]^{Q\quad\,}&\mathscr{C}[\Sigma^{-1}]\ar@^{-->}[d]^{\epsilon}\\
				&\mathscr{B}&.
			}
		}
	\end{propsn}
		\begin{proof}
			Supongamos $q:\mathscr{C}\to \mathscr{B}$ es un funtor tal que 
			\begin{itemize}
				\item[a)] $\forall \sigma\in \Sigma$ \quad $q(\sigma)$ es iso.
				\item[b)] $\forall f:\mathscr{C}\to \mathscr{A}$ tal que $F(\sigma)$ es iso $\forall \sigma\in \Sigma$, $\exists ! \bar{F}:\mathscr{B}\to\mathscr{A}$
				tal que $\bar{F}\circ q=F$.\\
			\end{itemize}
			
			Como $Q:\mathscr{C}\to\mathscr{C}[\Sigma^{-1}]$ es tal que $Q(\sigma)$ es iso $\forall \sigma \in \Sigma$, entonces por 
			hipótesis $\exists !\,\epsilon_0:\mathscr{B}\to \mathscr{C}[\Sigma^{-1}]$ tal que $\epsilon_0\circ q=Q$, es decir, el siguiente diagrama conmuta:\\
			\centerline{
				\xymatrix{
					\mathscr{C}\ar[dr]_{Q}\ar[r]^q&\mathscr{B}\ar@^{-->}[d]^{\epsilon_0}\\
					&\mathscr{C}[\Sigma^{-1}]&.
				}
			}
			Ahora, como $Q$ es funtor de localización y $q(\sigma)$ es iso $\forall \sigma\in \Sigma$, entonces por definición 
			$\exists !\,\epsilon:\mathscr{C}[\Sigma^{-1}]\to \mathscr{B}$ tal que $\epsilon\circ Q=q$. Así se tiene el siguiente diagrama:\\
			\centerline{
				\xymatrix{
					&\mathscr{C}[\Sigma^{-1}]\ar[d]^{\epsilon}\\
					\mathscr{C}\ar[ur]^Q\ar[dr]_Q\ar[r]^q&\mathscr{B}\ar@^{-->}[d]^{\epsilon_0}\\
					&\mathscr{C}[\Sigma^{-1}]&.
				}
			}
			
			En particular $\epsilon_0\epsilon$ es un funtor, y es tal que $\forall \sigma\in \Sigma,$\quad $\epsilon_0\epsilon(\sigma)$ es un isomorfismo. Así por ( L2 )
			sobre el funtor de localización $Q$, se tiene que $\epsilon_0\epsilon$ es único, pero $1_{\mathscr{C}[\Sigma^{-1}]}$ es un funtor con la misma
			propiedad ( pues $\sigma$ es iso para cada $\sigma \in \Sigma$  ) por lo tanto $\epsilon_0\epsilon=1_{\mathscr{C}[\Sigma^{-1}]}$ y analogamente
			$\epsilon\epsilon_0=1_{\mathscr{B}}$.\\
		\end{proof}
\end{document}